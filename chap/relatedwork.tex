\section{Related Works}%各个方向全都加入,例如安全分析
\label{sec:related}
SSO was first proposed in **. Now, the typical SSO standards include Kerberos, SAML, OAuth, OIDC and CAS, which has been adopted and implemented by Google, Facebook, and Twitter.
In 2014, Chen et al.\cite{ChenPCTKT14} concludes the problems developers may face to in using sso protocol. It describes the requirements for authentication and authorization and different between them. They illustrate what kind of protocol is appropriate to authentication. And in this work the importance of secure base for token transmission is also pointed.


\textbf{Various attacks were proposed for the impersonation attack and identity injection, by breaking the confidentiality, integrity and binding of identity proof.} ***
Besides of OAuth 2.0 and OpenID Connect 1.0, Juraj et al.\cite{SomorovskyMSKJ12} find XSW vulnerabilities which allows attackers insert malicious elements in 11 SAML frameworks. It allows adversaries to compromise the integrity of SAML and causes different types of attack in each frameworks.

\textbf{The SSO is also adopted in the mobile application}, for example, ** ** and ** provide the mobile SSO service. \textbf{However, new attacks were found,} as the mobile applications fails to ensure the  confidentiality, integrity and binding. ****

\textbf{The comprehensive formal security Analysis were performed on SAML, OAuth and OIDC.}  In 2016, Daniel et al.\cite{FettKS16} conduct comprehensive formal security Analysis of OAuth 2.0. In this work, they illustrate attacks on OAuth 2.0 and OpenID Connect. Besides they also presents the snalysis of OAuth 2.0 about authorization and authentication properties and so on. Other security analysis\cite{WangCW12}\cite{ZhouE14}\cite{WangZLG16}\cite{YangLLZH16}\cite{WangZLLYLG15} on SSO system concludes the rules SSO protocol must obey with different manners.


\textbf{As suggested in NIST SP800-63C~\cite{NIST2017draft}, user's privacy should be protected in SSO systems, which is partially achieved in the SSO standards, BrowerID and SPRESSO.}
The user's privacy protection~\cite{NIST2017draft} in SSO systems includes 1) the user should be able to control the range of the attributes exposed to the RP, 2) multiple RPs should fail to link the user through collusion, 3) IdP should fail to obtain the trace of RPs accessed by a user. The first property is satisfied in most SSO systems. For example, in OAuth and OIDC, IdP exhibits the  attributes requested by the RP and sends the attributes to the RP only when the user has provided a clear consent, which may also minimize the exposed attributes as the user may disagree to provide partial attributes.
BrowserID\cite{BrowserID}\cite{FettKS14} is a user privacy respecting SSO system proposed by Molliza. BrowserID allows user to generates asymmetric key pair and upload its public to IdP. IdP put user's email and public key together and generates its signature as user certificate (UC). User signs origin of the RP with its private key as identity assertion (IA). A pair containing a UC and a matching IA is called a certificate assertion pair (CAP) and RP authenticates a user by its CAP. But UC contains user's email so that RPs are able to link a user's logins in different RPs.
SPRESSO\cite{SPRESSO} allows RP to encrypt its identity and a random number with symmetric algorithm as a tag to present itself in each login. And token containing user's email and tag signed by IdP is also encrypted by a symmetric key provided by RP. During parameters transmission a third party credible website is required to forward important data. As token contains user's email, RPs are able to link a user's logins in different RPs.


\textbf{Anonymous SSO scheme is proposed to hide the user's identity to both the IdP and RPs, which can only be applied to the services that do not need the user's unique identifier.} 
Anonymous SSO schemes prevents the IdP from obtaining the user's identity for RPs who do not require the user's identity nor PII, and just need to check whether the user is authorized or not. These anonymous schemes, such as the anonymous scheme proposed by Han et al.~\cite{HanCSTW18}, allow user to obtain a token from IdP by proving that he/she is someone who has registered in the Central Authority based on  Zero-Knowledge Proof. RP is only able to check the validation of the token but unable to identify the user.
In 2018, Han et al.\cite{HanCSTW18} proposed a novel SSO system which uses zero knowledge to keep user anonymous in the system. A user is able to obtain a ticket for a verifier (RP) from a ticket issuer (IdP) anonymously without informing ticket issuer anything about its identity. Ticket issuer is unable to find out whether two ticket is required by same user or not. The ticket is only validate in the designated verifier. Verifier cannot collude with other verifiers to link a user's service requests. Same as the last work, system verifier is unable to find out the relevance of same user's different requests so that it cannot provide customization service to a user. So this system is not appropriate for current web applications.
In 2010, Han et al.\cite{HanMSY10} proposed a dynamic SSO system with digital signature to guarantee unforgeability. To protect user's privacy, it uses broadcast encryption to make sure only the designated service providers is able to check the validity of user's credential. User uses zero-knowledge proofs to show it is the owner of the valid credential. But in this system verifier is unable to find out the relevance of same user's different requests so that it cannot provide customization service to a user. So this system is not appropriate for current web applications.
In 2013, Wang et al. proposed anonymous single sign-on schemes transformed from group signatures. In an ASSO scheme, a user gets credential from a trusted third party (same as IdP) once. Then user is able to authenticate itself to different service providers (same as RP) by generating a user proof via using the same credential. SPs can confirm the validity of each user but should not be able to trace the user’s identity.
