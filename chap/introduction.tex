\section{Introduction}
\label{sec:intro}


%第一段:
%SSO的特点
%SSO的现状
%当前主流的SSO协议
To maintain each user's profile and provide individual services, each service provider needs to identify each user, which requires the users to be authenticated at multiple online services repeatedly. 
Single Sign-On (SSO) systems enable users to access multiple services (called relying parties, RP) with the single authentication performed at the Identity Provider (IdP). With SSO system deployed, a user only needs to maintain the credential of the IdP, who offers user's attributes (i.e., identity proof) for each RP to accomplish the user's identification. 
%Once a user has been authenticated by the IdP, he/she enjoys the services of the RP after the user has granted its attributes access permission to the corresponding RP.
%with just the simple permission granting. 
SSO system also brings the convenience to RPs, as the risks in the users' authentication are shifted to the IdP, for example, RPs don't need to consider the leakage of users' credentials. 
Therefore, SSO systems are widely deployed and integrated. 
%their own authentication systems, 
%实际数据
The survey on the top 100 websites demonstrates that 24 websites (e.g., Google, Facebook and Twitter) serve as the IdP while 63 websites integrate the SSO service.

%第二段:
%SSO 引入新的隐私问题
%IdP知道用户登录哪个RP
%RP之间可以合谋知道同一个用户登录哪些RP
However, SSO systems introduces new privacy issues in authentication which exposes the users' login traces. More detailedly, IdPs always know which RP a user accesses and RPs may link the same user who logs in each RPs, which can be easily avoided in traditional authentication systems by using different usernames for multiple services. 

%第三段:
%google and facebook的负面新闻
Currently the IdPs widely accepted by users, such as Google and Facebook are the large enterprise who has already held the significant quantity of users' data. For example, Facebook collects user data, such as where you live, your age, gender, level of education, employment details, language and so on, used for commercial purpose or something possibly even worse, such as political purpose. It is reported in 2016 the company Cambridge Analytica utilized the 50 million people's profiles leaked by Facebook to build the portrait of voters, in order to target them with personalised political advertisements. Google and Facebook seem to become the real Mr. Know It All, as they know who you are, where you live, what you interest in and so on, as long as you have to use the service provided by them. However, what is even worse is that they apparently are interested in what you have done in other applications without of their control. For example, Google would like to offer \$20 gift card for those people who are willing to accept the Screenwise Meter (mobile app and web extension) which allows Google to see what you are doing in other applications. However, with Google Identity service (Google's SSO system), it has the ability to surveil what applications people accessed even without any additional payment, which is to be used to draw their portraits. 



One basic requirement of SSO systems is the security, which includes two aspects: 1) the attacker should not be able to access the honest RP with the honest users' identity; 2) the identity injection will never succeed, that is, the attacker should not be able to make the honest user access the RP with an incorrect identity. Plenty of works are proposed for the security of SSO systems. 
Firstly, various standards, e.g., OAuth 2.0~\cite{rfc6749}, SAML~\cite{SAML} and OpenID Connect (OIDC)~\cite{OpenIDConnect}, are proposed to formalize the handling at each entity (i.e.,  the user, RP and IdP)  and the information exchanges between the entities. 
Secondly, the standards, SAML, OAuth and OIDC, are formally analyzed, for example, a general Dolev-Yao style web model is proposed for the web infrastructure~\cite{webmodel} and adopted to analyze the security of OAuth and OIDC~\cite{FettKS16}~\cite{FettKS17}.
Moreover, the typical implementations of SSO systems, e.g. Google, Facebook, Twitter and the corresponding RPs, are systematically analyzed~\cite{WangCW12}~\cite{FettKS16}~\cite{ZhouE14}~\cite{WangZLG16}, which makes  the security of SSO systems improved significantly.

%整理逻辑
%Although the security of SSO protocols and implementations has been systematically analyzed and user's privacy is considered in the design of SSO protocols, user's privacy has not been protected well yet. 

%Security and privacy are important in SSO systems. The security of SSO protocols and implementations are systematically analyzed~\cite{WangCW12}~\cite{FettKS16}~\cite{ZhouE14}~\cite{WangZLG16}. SAML, OAuth and OIDC are formally analyzed based on the web infrastructure model. The typical SSO solutions, e.g. Google, Facebook and Twitter, has been analyzed and the security has been improved significantly.

The other important requirement of SSO systems is the privacy. As suggested in NIST SP800-63C~\cite{NIST2017draft}, in SSO systems, 1) the user should be able to control the range of the attributes exposed to the RP, 2) multiple RPs should fail to link the user through collusion, 3) IdP should fail to obtain the trace of RPs accessed by a user. The first two properties are satisfied in the popular SSO systems. For example, in OAuth and OIDC, IdP exhibits the  attributes requested by the RP and sends the attributes to the RP only when the user has provided a clear consent, which may also minimize the exposed attributes as the user may disagree to provide partial attributes. To prevent a possible correlation among users from multiple RPs, a Pairwise Pseudonymous Identifier (PPID) is suggested to be generated by the IdP for the user in each RP, which requires that  the user's identifier in one RP should never be the same with or derivable from the ones of other RPs.

%As for the privacy of SSO systems, NIST SP800-63C~\cite{NIST2017draft} issues following informative advisement for federation of digital identity: (1) SSO systems should protect users from being tracked and profiled which means that: a. IdP is unable to know which RP a user has logged in; b. RP is also unable to know its user's identity in other RPs; (2) IdP should make users' personally identifiable information (PII) accessible only with user's consent; (3) IdP should enable the data exposed to an RP to be minimized. Current SSO protocols adopts several measures to protect users' privacy. For example, in OpenID Connect, IdP sends the user's PII to the RP only when the RP has obtained the user's consent for certain attributes, and provides the access logs for the user to monitor who accessed his/her PII. 


However, in widely deployed SSO systems, IdP knows which RP the user logs in, which reflects the service that user accesses and may be analyzed for various purposes, e.g., profiling and targeted advertising. In addition to the potential commercial purpose, exposing the identifier of accessed RP to the IdP, is required  for security consideration in existing SSO systems~\cite{ChenPCTKT14}. Firstly, the identity proof should only be sent to the correct RP, which prevents the adversary from performing the impersonation attack with the leaked identity proof. Secondly, the identity proof should be bound with a specific RP and user, which ensures the identity proof is only valid in the certain RP, and avoids the misuse of identity proof, for example, the adversary fails to  use the identity proof for a corrupted RP to  access another RP on behalf of the victim user.

%However, no existing SSO protocols is able to protect users from being tracked and profiled absolutely. Current widely adopted SSO protocols (e.g., OpenID Connect) protect the user from a possible correlation among RPs by using Pairwise Pseudonymous Identifier (PPID)~\cite{OpenIDConnect}. 
%But they are not designed to protect user from being tracked by IdP. For security consideration a token in OpenID Connect for authentication issued by IdP should be bound with specific RP and user. It has been discussed in ~\cite{ChenPCTKT14}. 
%Some work, such as~\cite{ChenPCTKT14} has issued that for security consideration the identity proof offered by IdP should be bound with specific RP and user so that RP's identity has to be provided to IdP. As a result IdP is always able to know which RP a user has logged in. 


%To guarantee the security, SAML, OAuth 2.0 and OpenID Connect are formally analyzed based on the web infrastructure model. And the typical SSO solutions, e.g. Google, Facebook and Twitter, has been analyzed and the security has been improved significantly. To protect user's privacy, for example, in OpenID Connect, IdP sends the user's personally identifiable information (PII) to the RP only when the RP has obtained the user's consent, and provides the access logs for the user to monitor who accessed his/her PII. Moreover, to protect the user from a possible correlation among RPs, IdP should provide each RP separate pseudonym for a user as the identifier. 
%单独
%There is, however, no existing popular SSO protocol being designed to protect user from being tracked by IdP.     

%In addition to security, the privacy has been well considered in the design of SSO protocols.
%For example, in OIDC, IdP sends the user's personally identifiable information (PII) to the RP only when the RP has obtained the user's consent,
 %and provides the access logs for the user to monitor who accessed his/her PII.
%Moreover, to protect the user from a possible correlation among RPs, IdP should use a pairwise pseudonymous %identifier as the user's identifier (i.e., \verb+sub+) returned to the RP.


%However, current SSO systems still allow the IdP to track the users .
%For security consideration, the ticket in OAuth and OIDC,  issued by the IdP should contain the identity of RP, to send the ticket or ticket index, the IdP should construct the rediect URL which contains the  RP URL.
%第三段
%现有研究
%现有研究的缺点
%能否提供统一的隐私要求

%匿名SSO打算放到related work中
%In addition to widely adopted SSO systems, various SSO schemes are proposed to protect user's privacy.  These protocols can be classified by their objective: (1) preventing the IdP from obtaining the user's identity and (2) 
%avoiding the IdP learning at which RP the user logins in. Anonymous SSO schemes belong to the first one and apply to the RPs who do not require the user's identity nor PII, and just need to check whether the user is authorized or not. These anonymous schemes, such as the anonymous scheme proposed by Han et al.~\cite{HanCSTW18}, allow user to obtain a token from IdP by proving that he/she is someone who has registered in the Central Authority based on  Zero-Knowledge Proof. RP is only able to check the validation of the token but unable to identify the user.
%These schemes, such as  allow users to access a service protected by a verifier without revealing their identities to the RPs.
%However, to provide the continuous and personalized service, RP needs to obtain the unique pseudonym for each user. In this case, the solutions belong to the the other case is more suitable, as the IdP doesn't know which RP the user accesses.


Two SSO systems (BrowserID~\cite{persona} and SPRESSO~\cite{SPRESSO}) are proposed to hide the user's accessed RPs from IdP, while ensuring the security of SSO systems simultaneously. In BrowserID, the user is responsible for sending the identity proof correctly and binding the proof with the correct RP using a newly generated private key, while the corresponding public key is bound with the email address by the IdP who fails to obtain the information of accessed RP. In SPRESSO, the identity proof is bound with an encryption of the RP's domain name by the IdP who knows the user's identity but not the plaintext of the RP's information, and sent to the exact RP by a newly introduced trusted entity (called FWD) who obtains the RP's domain name but not the identity of the user. 

%In BrowserID, the identity proof sent by the user to the RP directly, contains two parts: one is generated by the IdP which binds the user's email address with a public key, and the other is a signature of the RP generated by the user with the corresponding private key. Therefore, IdP in BrowserID fails to obtain the information of RP. In SPRESSO, RP encrypts its domain name (and a random nonce) with a symmetric key as the identifier  to obtain the identity proof, which makes others fail to infer the RP from the identifier, and a new entity (FWD) is introduced to relay the identity proof from the IdP to the RP. Therefore, for one identity proof in SPRESSO, IdP only knows the user's identity  while the FWD only knows the RP.

BrowserID and SPRESSO are both redesigns of SSO systems, and therefore incompatible with existing widely deployed SSO systems (e.g., OAuth, OIDC and SAML). Moreover, the new SSO systems require a  complicated, formal and thorough security analysis of both the designs and various implementations. As shown in ~\cite{BrowserID}, vulnerabilities have been found in the implementation of BrowserID. More importantly,  in order to provide the same identity to the RP in the multiple logins of a user, both BrowserID and SPRESSO use the email address as the identity, which makes the user linkage (from multiple RPs) possible.

%Vairous SSO protocols are proposed to hide the users' accessing to RPs from the IdP. In the BrowserID system~\cite{persona}, a user firstly generates a asymmetric key pair. The IdP authenticates the user and offers a user certificate (UC) which contains the user's email address with the user's public key. User is to sign the origin of the RP with the corresponding private key as the identity assertion (IA). RP is able to identify the user by UC and IA. 
%the user sends RP the user certificate (i.e. combing the user's email address with the user's public key) and identity assertion (i.e., the origin of the RP signed by the user with the corresponding private key) to complete the authentication, and is formally analyzed and found severe attacks~\cite{BrowserID}.
%SPRESSO~\cite{SPRESSO} is designed based on the standard HTML5 and web features, and formally analyzed based on the expressive Dolev-Yao style model of the web infrastructure~\cite{webmodel}. But for RPs to identify a user, both BrowserID and SPRESSO need to provide user's real identity (or a constant pseudonym) to each RP. Communication among multiple RPs would allow RPs to profile the user.  
%If a user logs in multiple RPs, these RPs are able to draw his/her trace. 


%Therefore there are no SSO protocols so far that protect users from tracking by both IdP and RPs simultaneously. NIST publication has issued that proxy is able to protect users' privacy in SSO system. It is defined that when the proxy can keep IdP and RP anonymous to each other and itself the proxy is called the triple blind proxy. But no existing proxy has achieved this goal. 
%that the triple blind proxy can protect users from being tracked, but there is no existing systems achieved this goal. 
%Moreover, the protocols protecting user's privacy (such as, SPRESSO and BrowserID) are quite distinct from the widely adopted protocols. There is to be a huge cost if IdP developers migrate their systems into a totally brand-new architecture.
%第四段
%IdP钓鱼网站问题
%转移到安全分析
%Besides tracking user, IdP in traditional SSO system is not expected to protect users from phishing attack either. Imagine that you visit \emph{phishingsite.com} which is established by a malicious opponent and allows user to log in using a Google account. Google SSO allows web application to direct user to Google's login web page and user need log in Google. However, \emph{phishingsite} directs you to a phishing site totally same as Google's web page instead of the real one. That is, while you accomplish the authentication, e.g., inputting your password, your password is to be stolen by malicious opponent. 

%Phishing attack in SSO systems has been considered by OAuth 2.0~\cite{rfc6749} and several papers~\cite{anti_phishing}~\cite{devil_phishing}~\cite{mechanism_phishing} are proposed to discuss it. There is, however, no existing schemes can be adopted to enhance the security of popular protocols, e.g., OpenID Connect. 
 

In this paper, we propose an extension (named PriOIDC) of existing widely adopted SSO system (i.e., OIDC), which preserves the systematically and thoroughly analyzed security, and achieves the fully privacy. That is, (1) the security design in OIDC is inherited to prevent the impersonation attack and  identity injection attack, (2) the privacy enhance mechanisms (e.g., the clear consent from the user and the PPID) are retained, (3) a new mechanism is introduced to hide the user's accessed RP from IdP. Unlike designing and deploying a new SSO systems, we only need to analyze the compliance of the new function (i.e., hiding the user's accessed RP) and the influence to the security introduced by the new mechanism. And, the deployment of PriOIDC only requires: (1) IdP provides a set of public parameters and generates the PPID with a newly provided algorithm, (2) RP integrates the SSO service with a new version software development kit (SDK) whose interface remains unchanged, (3) the user installs an extension to access RP with full privacy anywhere as no persistent storage is required in the user side.

%第⑤段:
%我们的目标
%建立一套新的单点登录系统,满足:
%1 满足NIST的隐私需求
%2 便于现有系统的迁移

%Therefore the goal of this work is to design a novel SSO system which provides the following features: 
%(1) it should protect users from being tracked by both IdP and multiple RPs at the same time; 
%(2) it should be convenient for developers to complete system migration from traditional SSO system.

To hide the user's accessed RP from IdP, PriOIDC avoids the potential leakage in the identity proof (i.e., id token in OIDC)  and the corresponding message transmission. Moreover, PriOIDC enables only the RP (and the user) to derive the user's unique identifier from different PPIDs,  which  allows the RP to provide the individual service with the unique identifier and avoids the user linkage as both the PPID and user's unique identifier in RP are different for various RPs. PriOIDC achieves these as follows: 
%which is distinct in each RP 
\begin{itemize}
\item A new algorithm is proposed to negotiate the RP's identifier between the user and RP for each login. Therefore, the RP's identifier in multiple id tokens are different, and IdP fails to infer RP's information in the generation of one or multiple id tokens. Moreover, neither RP nor the user may control the generated identifier, which avoids the misuse of the id token. The detailed analysis is provided  in Section~\ref{sec:analysis}.
\item A browser extension is introduced to transmit the messages (i.e.,  request and response) related with the id token. IdP fails to infer the RP's information through the network traffic, and the extension ensures only the correct RP receives the id token.
\item A new generation algorithm of PPID is provided, which makes the PPIDs for one user in one RP  indistinguishable  from others (e.g., different users in different RPs), while only the RP (and the user) has the trapdoor to derive the unique identifier from different PPIDs for one user in one RP.
\end{itemize}

%加入基本的解决方案 challenges
%\textbf{Challenge}. It is discussed that a user's identity proof provided by IdP should be bound with specific RP and the user\cite{rfc6749}\cite{ChenPCTKT14}\cite{WangZLG16}. Moreover, the proof should be linked with a unique RP id and user id. As for current widely adopted SSO protocols, for example, OpenID Connect provides an id token for user which contains RP's id (named \verb+aud+) and user's id (named \verb+sub+). Firstly RP sends its \verb+aud+ to IdP and IdP finds out the \verb+sub+ of user solely correlated with the \verb+aud+. Then IdP generates the id token with the \verb+aud+ and \verb+sub+ and sends it RP. RP is going to identify the user by \verb+sub+. So while RP doesn't provide it's identity to IdP, IdP is unable to find out the correlated \verb+sub+ so that RP cannot identify the user. 
%The \verb+aud+ is the identity of RP user offers to IdP and \verb+sub+ is the user's PPID generated by IdP. To keep RP anonymous at IdP the aud should be random for each authentication. However, in consideration of privacy \verb+sub+ should solely correlate a unique \verb+aud+. It means random \verb+aud+ is to result in random \verb+sub+ which prevents RP from offering continuous and personalized service to user. 
%The challenge is how to identify a specific user without exposing the identity of RP to IdP. To achieve this goal, we propose the new id generating algorithm of RP and user for OpenID Connect. It enables RP to generate a random \verb+aud+ for each authentication and IdP generates \verb+sub+ with this pseudonym. RP is able to translate the random \verb+sub+ into a constant user identity.   

%第六段
%我们的贡献
%提出协议
%考虑能否根据模型进行分析
%实现原型系统
The main contributions of PriOIDC are as follows:
\begin{itemize}
\item We propose a pratical extension for OIDC, which inherits the systematically and thoroughly analyzed  security and privacy mechanisms of OIDC, and achieves the full privacy for users by hiding the accessed RPs from IdP.
\item We developed the prototype of PriOIDC. The evaluation demonstrates the effectiveness and efficiency of PriOIDC. We also provide a systematic analysis of PriOIDC to prove that PriOIDC introduces no degradation  in the security of OIDC.
\end{itemize}

%\textbf{Contributions.} 
%1. We have designed a practical Enhanced OpenID Connect 1.0 Protocol called PriOIDC which is the first SSO protocol that protect users from being tracked by both IdP and multiple RPs at the same time. In this system IdP only knows a user wants to log in an RP but never knows which RP it is. And IdP offers a user separate pseudonym in different RPs as user's identifier so that multi-RP collusion cannot deduce the user's login trace either. 2. We have provided the prototype, and its overhead is proved to be modest (less than 50ms of each login on average). 

%第七段
%文章结构
The rest of this paper is organized as follows. We introduce the background and the threat model in Sections~\ref{sec:back} and~\ref{sec:overview}. Section~\ref{sec:protocol} describes the design  and details of PriOIDC. A systematical analysis is presented in Section~\ref{sec:analysis}. We provide the implementation specifics and evaluation in Section~\ref{sec:evaluation}, then introduce the related works in Section~\ref{sec:related}, and draw the conclusion finally. 

%This paper is organized as follows: Section~\ref{sec:back} provides the knowledge of OpenID Connect. Section~\ref{sec:overview} gives the overview of this scheme and its attack surface. Section~\ref{sec:protocol} describes the construction and details of this scheme. Section~\ref{sec:analysis} provides the detailed analysis of the new scheme. Section~\ref{sec:evaluation} offers a performance evaluation of our prototype system. Section~\ref{sec:related} discusses the related works. In Section~\ref{sec:conclusion}, a conclusion is given.



\begin{comment}
%These protocols are classified into two classes according to the objective:
%(1) preventing the RP obtaining the user's identity and (2) avoiding the IdP learning at which RP the user logins in.
Anonymous SSO authentications schemes belonging to the first one and apply to the RPs who do not require the user's identity nor PII, and just  need to check whether the user is authorised or not. These schemes allow users to access a service protected by a verifier without revealing their identities to the RPs.

However, to provide the continuous and preconized service, RP needs to obtain the unique pseudonym for each user.
In this case, the solutions belong to the class 2 is more suitable, as the IdP doesn't know which RP the user access while the RP only knows the user's pseudonym but not the real identity.
Vairous SSO protocols are proposed to hide the users accessed RPs from the IdP. 
In the BrowserID system~\cite{persona}, the user sends RP the user certificate (i.e. combing the user's email address with the user's public key) and identity assertion (i.e., the origin of the RP signed by the user with the corresponding private key) to complete the authentication, and is formally analyzed and found severe attacks~\cite{BrowserID}.
SPRESSO~\cite{SPRESSO} is proposed based on the standard HTML5 and web features, and formally analyzed based on the expressive Dolev-Yao style model of the web infrastructure~\cite{webmodel}.





For security consideration, the ticket issued by the IdP should contain the identity of RP, to send the ticket or ticket index, the IdP should construct the rediect URL which contains the  RP URL.

%大量工作来提升协议的安全性,包括1.协议本身的安全性;2. 具体实现的安全性。
The security of SSO systems is improved by .
The OAuth 2.0 and OpenId-Connect protocol are formally analyzed under the web model.
The implementations have been evaluated and various vulnerabilities are found which
%Authentication is the first line to protect the user information,

%SSO引入了新的隐私问题,目前已有多项工作1. 协议规范要求IdP在向RP传输PII时,需要得到用户明确授权;2.用户在不同RP使用的不同的标识,防止串联 3.匿名SSO,防止RP获取用户信息


%但是SSO引入了一个明显的问题,即IdP将获取用户访问的RP信息,从而能够***


%本文提出了一种匿名方案,通过集成***,实现了。可以描述该工作的难点





A Single Sign-On (SSO) system contains Identity Provider (IdP), Relying Parties (RP) and users. It allows users to log in the IdP once and then user can use RPs' service without extra login. Currently internet services have been an essential part of people's life. To provide their users with personalized services, many web application developers manage to authenticate users. Traditional authentication system requires users to keep individual accounts and passwords.

%SSO引入了新的安全问题 However, SSO introduces new risk to the user's privacy.
Introduce the flaws of current sso system. All the popular sso protocols can't hide the login information , who login in which web application. And even the specification of some protocols advise the IdP should keep the user identity unique in each RP, but may sso service providers haven't confirm the rule, such as Google.

%现有的缓解方法
Introduce what we want about the sso system. When the user logs in a RP with the sso service, the IdP will get nothing can link the user an the RP. And even the multi-RP collusion cannot deduces the user's login trace.

%但是现有解决方法还不够,引入我们的问题
Our contributions:

%我们方法的基本原理

\end{comment}