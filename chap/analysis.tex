\section{Analysis}
\label{sec:analysis}
In this section, we firstly prove the privacy of UPRESSO, i.e., avoiding the identity linkage at the collusive malicious RPs,
and preventing the curious IdP from inferring the user's accessed RPs.
Then, we prove that UPRESSO does not degrade the security of SSO systems by comparing it with OIDC, which has been formally analyzed in~\cite{FettKS17}.


\subsection{Security}
\label{subsec:security}
UPRESSO protects the user's privacy without breaking the security. That is, UPRESSO still prevents the malicious RPs and users from breaking the user identification, receiver designation and integrity.

In UPRESSO, all mechanisms for integrity are inherited from OIDC. The IdP uses the un-leaked private key $SK_{ID}$ to prevent the forging and modification of identity proof. The honest RP (i.e., the target of the adversary) checks the signature using the public key $PK_{ID}$, and only accepts the elements protected by the signature.

For the requirement of receiver designation of identity proof, UPRESSO inherits the same idea (trusted transmission of identity proof and binding the identity proof with specific RP) from OIDC.


%Two-step check: 
%原始的OIDC系统中,由IdP验证请求中的RP ID和endpoint是否是在IdP进行过注册,防止将对应某个RP的identity proof发送给adversary。
%由于在UPRESSO中,IdP不知道RP的身份,所以无法进行验证,需要将验证的过程交给user完成。
%因为user不保存RP ID和endpoint的对应信息,所以需要通过IdP签发的RP Certificate进行验证。The first step要求user验证Certificate的正确性,并从Certificate中获得ID_RP和endpoint的信息;the second step要求用户验证identity proof中的PID_RP与Certificate中ID_RP的对应,并且检查重定向地址保证identity proof发送给对应的RP。
For example, in UPRESSO TLS, a trusted user agent and the checks are also adopted to guarantee the trusted transmission. TLS avoids the leakage and modification during the transmitting. The trusted agent ensures the identity proof to be sent to the correct RP based on the endpoint specified in the $Cert_{RP}$. The  $Cert_{RP}$ is protected by the signature with the un-leaked private key $SK_{Cert}$, ensuring it  will never be tampered with by the the adversary. For UPRESSO, the check at RP's information is exactly the same as OIDC, that is, checking the RP identifier and endpoint in the identity proof request with the registered ones, preventing the adversary from triggering the IdP to generate an incorrect proof and transmitting it to the incorrect RP. However, the user in UPRESSO performs a two-step check instead of the direct check based on the $ID_{RP}$ in OIDC. Firstly, the user checks the correctness of $Cert_{RP}$ and extracts  $ID_{RP}$ and the endpoint. In the second step, the user checks whether the RP identifier in identity proof request is the $PID_{RP}$ negotiated for this authentication based on the $ID_{RP}$ and the endpoint is also the one in $Cert_{RP}$. This two-step check also ensures the identity proof for the correct RP ($ID_{RP}$) is sent to correct endpoint (one specified in $Cert_{RP}$).

The mechanisms for binding are also inherited from OIDC. The IdP binds the identity proof with $PID_{RP}$, and the correct RP checks the binding by comparing the $PID_{RP}$ with the cached one.
%, and provides the service  to the $Account$ based on $PID_U$.

\vspace{1mm}\noindent\textbf{Receiver designation.} UPRESSO binds the identity proof with $PID_{RP}$, instead of IdP chosen RP identifier for each RP assigned by IdP in OIDC. However, the adversary (malicious users and RPs) still fails to  make one identity proof 
%(or its transformation) 
accepted by another honest RP. As the honest RP only accepts the valid identity proof for its fresh negotiated $PID_{RP}$, we only need to ensure one $PID_{RP}$ (or its transformation) never be accepted by the other honest RPs.
\begin{itemize}
\item $PID_{RP}$ is unique in one IdP. The honest IdP checks the uniqueness of $PID_{RP}$ in its scope during the dynamic registration, to avoid one $PID_{RP}$ (in its generated identity proof) corresponding to two or more RPs. Otherwise, there is hardly to be the conflicts between different login flows. We assume that the dynamically registered RP identifier is valid in 5 minutes (also the valid ticket window) and there are 1 billion ($2^{30}$) login requests during this period, as the chosen prime number p is 2048-bit long, the probability of conflict existing is about $1-\prod_{i=1}^{2^{30}}(2^{2048-i}/2^{2048})$, which is almost 0. Moreover, the mapping of $PID_{RP}$ and IdP globally unique. The identity proof contains the identifier of IdP (i.e., \verb+issuer+), which is checked by the correct RPs. Therefore, the same $PID_{RP}$ in different IdPs will be distinguished.
%\item The $PID_{RP}$ in the identity proof is protected by the signature generated with $SK_{ID}$. The adversary fails to replace it with a transformation without invaliding the signature.
\item The correct RP or user prevents the adversary from manipulating the $PID_{RP}$. For extra benefits, the adversary can only know or control one entity in the login flow (if controlling the two ends, no victim exists). The other honest entity provides a random nonce ($n_U$ or $n_{RP}$) for $PID_{RP}$. The nonce is independent from the ones previously generated by itself  and the ones generated by others, which prevents the adversary from controlling the $PID_{RP}$. Moreover, The $PID_{RP}$ in the identity proof is protected by the signature generated with $SK_{ID}$. The adversary fails to replace it with a transformation without invaliding the signature.

\end{itemize}

\vspace{1mm}\noindent\textbf{User identification.} UPRESSO ensures the identification by binding the identity proof with $PID_U$  in the form of $PID_{RP}^{ID_U}$, instead of a randomly IdP generated unique identifier. However, the adversary still fails to login at the honest RP using a same $Account$ as the honest user. Firstly, the adversary fails to  modify the $PID_U$ directly in the identity protected by $SK_{ID}$. Secondly, the malicious users and RPs fail to trigger the IdP generate a wanted $PID_U$, as they cannot (1) obtain the honest user's $PID_U$ at the honest RP; (2) infer the $ID_U$ of any user from all the received  information (e.g., $PID_U$) and the calculated ones (e.g., $Account$); and (3) control the $PID_{RP}$ with the participation of a correct user or RP.

\vspace{1mm}\noindent\textbf{Protection conducted by user agent.} The design of UPRESSO makes it immune to some existing known attacks~\cite{FettKS16} (e.g., CSRF, 307 Redirect) on the implementations. The Cross-Site Request Forgery (CSRF) attack is  usually exploited by the adversary to perform the identity injection. However, in UPRESSO, the honest user logs  $PID_{RP}$ and one-time endpoint in the session,  and performs the checks before sending the identity proof to the RP's endpoint, which prevents the CSRF attack. The 307 Redirect attacks~\cite{FettKS16} is due to the implementation error at the IdP, i.e. returning the incorrect status code (i.e., 307), which makes the IdP leak the user's credential to the RPs during the redirection. In UPRESSO, the redirection is intercepted by the trusted user agent which removes these sensitive information. 
%In the IdP Mix-up attack, the adversary works as the IdP to collect the makes \verb+access token+ and \verb+authorization code+ (identity proof in OAuth 2.0) from the victim RP. Same as OIDC, UPRESSO includes the \verb+issuer+ in the identity proof (protected by the $SK_{ID}$), avoiding the victim RP to send the sensitive information to the IdP. The user established the TLS connection with RP and IdP, which avoids the Man-in-middle attack.

\subsection{Privacy}
\label{subsec:privacy}
\noindent{\textbf{Curious IdP.}} The curious IdP might be interested in the user accessed RP or infer the correlation of RPs in two or more login flows by performing the analysis on the content and timing of received messages. However, it fails to obtain the user's accessed RPs directly, nor classifies the accessed RPs for RP's information indirectly.
%The curious IdP can only perform the analysis on the content and timing of received messages, however it fails to obtain the user's accessed RPs directly, nor infer classifies the accessed RPs for RP's information indirectly.
\begin{itemize}
  \item The curious IdP might be interested in the RP's identity but fails to derive RP's identifying information (i.e., $ID_{RP}$ and correct endpoint) through a single login flow. IdP only receives $PID_{RP}$ and one-time endpoint, and fails to infer the $ID_{RP}$ from $PID_{RP}$ without the  trapdoor $t$
%  due to hardness of solving discrete logarithm,
  or the RP's endpoint from the independent one-time endpoint.
  \item It also might try to infer the correlation of RPs in two or more login flows but fails achieve the relationship between the $PID_{RP}$s. The secure random number generator ensures the random for generating $PID_{RP}$ and the random string for one-time endpoint are independent in multiple login flows. Therefore, curious IdP fails to classify the RPs based on $PID_{RP}$ and one-time endpoint.
  %It also might try to achieve the relationship between the $PRPID$s but fails to infer the correlation of RPs in two or more login flows from a single user or multiple users. The secure random number generator ensures the random for generating $PRPID$ and the random string for one-time endpoint are independent in multiple login flows. Therefore, curious IdP fails to classify the RPs based on $PRPID$ and one-time endpoint.
  \item It even fails to obtain the correlation of RPs through analyzing the timing of received messages. IdP fails to map user's accessed RP in the identity proof to the origin of dynamic registration based on  timing, as both the dynamic registration and the identity proof request are sent by the user instead of the RP.
\end{itemize}

\noindent{\textbf{Malicious RP.}} The malicious RPs may attempt to link the user passively by combining the $PID_U$s received by the collusive RPs, or actively by tampering with the provided elements (i.e., $Cert_{RP}$, $Y_{RP}$ and $PID_{RP}$). However, these RPs still fail to obtain the $ID_U$ directly, or trigger the IdP to generate a same or derivable $PID_U$s.
%The malicious RPs may attempt to link the user passively by combining the PPIDs received by the collusive RPs, or actively by tampering with the provided elements (i.e., $Cert_{RP}$, $Y_{RP}$ and $PRPID$). However, these RPs still fail to obtain the user's unique identifier directly, or trigger the IdP to generate a same or derivable PPIDs.
\begin{itemize}
\item A single RP might try to find out the $ID_U$ presenting the unchanged user identity but fails to infer the user's unique information (e.g., $ID_U$ or other similar ones) in the passive way. The $PID_U$ is the only element received by RP that contains the user's unique information. However, RP fails to infer (1) $ID_U$ (the discrete logarithm) from $PID_U$, due to hardness of solving discrete logarithm; (2) or $g^{ID_U}$ as the $r$ in $ID_{RP}=g^r$ is only known by IdP and never leaked, which prevents the RP from calculating $r^{-1}$ to transfer $Account=ID_{RP}^{ID_U}$ into  $g^{ID_U}$.
\item A single RP fails to actively tamper with the messages to make $ID_U$ leaked. The modification of  $Cert_{RP}$ will make the signature invalid and be found by the user. The malicious RP fails to manipulate  the calculation of $PID_{RP}$ by providing an incorrect $Y_{RP}$ as another element $n_U$ is  controlled by the user. Also, the malicious RP fails to make an incorrect $PID_{RP}$ (e.g., 1)  be used for $PID_U$, as the honest IdP only accepts a primitive root as the $PID_{RP}$ in the dynamic registration. The RP also fails to change the accepted $PID_{RP}$ in Step 11 in Figure~\ref{fig:process}, as the user checks it with the cached one.
\item Two or more RPs might try to find out whether the $Account$s in each RP are belong to one user or not but fail to link the user in the passive way. The analysis can only be performed based on $Account$ and $PID_U$. However, the $Account$ is independent among RPs, as the $ID_{RP}$ chosen by honest IdP is random and unique and the $PID_U$s are  also independent due to the unrelated $PID_{RP}$.
\item Two or more RPs also might lead IdP to generate the $PID_U$s same or  derivable into same $Account$ in each RP. Since the $PID_U$ is generated related with the $PID_{RP}$, corrupted RPs might choose the related $n_{RP}$ to correlate their $PID_{RP}$, however, the $PID_{RP}$ is also generated with the participation of $n_{U}$, so that RP does not have the ability to control the generation of $PID_{RP}$. Moreover, corrupted RPs might choose the same $ID_{RP}$ to lead the IdP to generate the $PID_U$ derivable into same $Account$, however, $ID_{RP}$ is verified by the user with through the $Cert_{RP}$, where the tampered $ID_{RP}$ is not acceptable to the honest user.
%achieve it in the active way, by attempting to make $PRPID$ correlative by manipulating $n_{RP}$ or use the same $RPID$. However, the random $n_u$ chosen by the honest user will ensure the independence of $PRPID$ to protect its own privacy.

\end{itemize}

Collusive RPs even fail to correlate the users based on the timing of users' requests, when the provided services are unrelated. For the related services, (e.g., the online payment accessed right after an order generated on the online shopping), the user may break this linking by adding an unpredicted time delay between the two accesses. The anonymous network may be adopted to prevent collusive RPs to classify the users based on IP addresses.


