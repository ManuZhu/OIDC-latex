\section{Implementation and Performance Evaluation}
\label{sec:implementation}
We have implemented the prototype of UPRESSO, and compared its performance with the original OIDC implementation and SPRESSO.

\subsection{Implementation}
We adopt SHA-256 to generate the digest, and  RSA-2048 for the signature in  the $Cert_{RP}$, identity proof and the dynamic registration response. We  choose a random 2048-bit strong prime as $p$, and the smallest primitive root (3 in the prototype)  of $p$ as $g$. The  $n_U$, $n_{RP}$ and $ID_U$  are 256-bit odd numbers, which provides equivalent security strength than RSA-2048~\cite{barkerecommendation}.

The implementation of IdP only need introduce the minimal modification of existing OIDC implementation. The IdP is implemented based on MITREid Connect~\cite{MITREid}, an open-source OIDC Java implementation certificated by the OpenID Foundation~\cite{OIDF}. %Alghough, OIDC standard specifies that RP's identifier should be generated by IdP in the dynamic registration, MITREid Connect allows the user to provide a candidate RP identifier to the IdP who checks the uniqueness, which simplifies the implementation of UPRESSO. 
In UPRESSO, we add 3 lines Java code for generation of $PPID$, 25 lines for generation of signature in dynamic registration, modify 1 line for checking the registration token in dynamic registration, while the calculation of $ID_{RP}$, $Cert_{RP}$,  $PID_U$, and the RSA signature is implemented using the Java built-in cryptographic libraries (e.g., BigInteger)

The user-side processing is implemented as a Chrome extension with about 330 lines JavaScript code and 30 lines  Chrome extension configuration files (specifying the required permissions, containing reading chrome tab information, sending the HTTP request, blocking the received HTTP response). The cryptographic calculation in $Cert_{RP}$ verification, $PID_{RP}$ negotiation, dynamic registration, is based on an efficient JavaScript cryptographic library  jsrsasign~\cite{jsrsasign}. 
%The Chrome extension clears the \verb+referer+ in the HTTP header, to avoid the RPs' URL leaked to the IdP. 
Moreover, the chrome extension needs to construct cross-origin requests to communicate with the RP and IdP, which is forbidden by the same-origin security policy as default. Therefore it is required to add the HTTP header \verb+Access-Control-Allow-Origin+ in the response of IdP and RP to accept only the request from the origin \verb+chrome-extension://chrome-id+ (\verb+chrome-id+ is uniquely assigned by the Google).
% UPRESSO adopts CORS to achieve this cross-origin communication. In details, we requires the RP and IdP to specify \verb+chrome-extension://chrome-id+ in the \verb+Access-Control-Allow-Origin+ field of its response header, which makes the request pass the permission checks at the browser. As \verb+chrome-id+ is unique assigned by the Google, no other (malicious) entity can perform the cross-origin communication.


We provide the SDK for RP to integrate UPRESSO easily. The SDK provides 2 functions:  %RP initial registration,
processing of the user's login request and  identity proof parsing. The Java SDK is implemented based on the Spring Boot framework  with about 1100 lines JAVA code. The cryptographic computation is completed through Spring Security library.
RP processing login request containing identifier negotiation and renewal in Figure~\ref{fig:process} and identity proof parsing containing account calculating in Figure~\ref{fig:process}.

%The user's login request implemented by SDK contains RP identifier negotiation and renewal in Figure~\ref{fig:process}; while identity proof parsing contains account calculating in Figure~\ref{fig:process}.






\begin{comment}
\subsection{Prototype Implementation}

\noindent\textbf{Cross-Origin Resource Sharing (CORS).} The chrome extension needs to construct cross-origin requests to communicate with the RP and IdP, which is forbidden by default by the same-origin security policy. UPRESSO adopts CORS to achieve this cross-origin communication. In details, we requires the RP and IdP to specify \verb+chrome-extension://chrome-id+ in the \verb+Access-Control-Allow-Origin+ field of its response header, which makes the request pass the permission checks at the browser. As \verb+chrome-id+ is unique assigned by the Google, no other (malicious) entity can perform the cross-origin communication.

\noindent\textbf{307 Redirect. }It has been discussed in~\cite{FettKS16} that IdP might redirect the user to the RP immediately after the user inputs the credentials. For example, the HTTP response to the user's POST message with \verb+username+ and \verb+password+ might be the redirection to RP carrying user's identity proof. That is, as long as the 307 status code is used for this redirection, the user's credentials are also transmitted to the RP. However, in UPRESSO the redirections are intercepted by the user agent and rebuild the HTTP GET request to RP or IdP which is unable to leak the POST data of the user.



\noindent\textbf{Cross-Site Request Forgery (CSRF). } The CSRF attack might lead the user to access the malicious url provided by the adversary's web page, through which the adversary might lead the honest user to upload the adversary's \verb+ID token+ to the RP. However, in UPRESSO the cross origin request should be repudiated by both RP and IdP excepted the request from the origin of the user agent, which is able to prevent the CSRF attack.



\end{comment}
