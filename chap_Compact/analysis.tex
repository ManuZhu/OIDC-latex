\section{Security Analysis}
\label{sec:analysis}
In this section, we propose the security properties of privacy-preserving SSO schemes and then give the proofs that UPPRESSO follows the security properties.
\subsection{Security Properties}
======= % ��һ���֣���ǰ����Щ�ظ�
\noindent\textbf{Authentication.}
Firstly, base on the existing security analysing work on SSO systems  \cite{ArmandoCCCT08, FettKS16, FettKS17, SomorovskyMSKJ12, WangCW12, ArmandoCCCPS13, ZhouE14, WangZLLYLG15, WangZLG16,YangLLZH16, MainkaMS16, MohsenS16, MainkaMSW17, YangLCZ18, YangLS17, ShiWL19} , we can summarize basic requirements of security SSO authentication.
\begin{itemize}
\item \textbf{Confidentiality. }Anyone who holds the identity token can prove its identity to the server, therefore, the identity token must be well protected to avoid leaking it to the adversary by neither being stolen during transmission, nor sent to the adversary party(e.g. IdP Mix Up attack).
\item \textbf{Integrity. }Only the IdP is able to generate a valid identity token, no other entity should be able to modify or forge it without being found. And, the honest RP should only accept the valid identity token.
\item \textbf{RP designation. }IdP issues the identity token for specific RP(s), so that only the target RP would accept the identity token from user. Otherwise, when a malicious receives a valid identity token from an honest user, it can log in to other RPs as this user with this identity token.
\item \textbf{User identification. }IdP should always provide the unique identifier for each user, and only the honest user can achieve an identity token representing this identifier.
%That is, for $\mathcal{F}_{ID_{RP} \mapsto PID_{RP}}$, $\mathcal{F}_{ID_{U} \mapsto PID_{U}}$ and $\mathcal{F}_{PID_{U} \mapsto Account}$ algorithms, the adversary cannot forge the identity token accepted by an RP, whose $Account$  is same as another honest user's in this RP.
\end{itemize}

\noindent\textbf{Privacy.}
The privacy-preserving SSO system requires that (a) the curious but honest IdP should not learn the user's visited RP. It means the following requirements must be satisfied.
\begin{itemize}
\item  IdP should always fail to derive RP's identity information (i.e., $ID_{RP}$ and real endpoint) through  a single login flow, and fail to distinguish whether the multiple login flows are from the same RP or not.
\item  RPs cannot infer a user's unique identifier (i.e., the $ID_U$), or find out whether the $Account$s in each RP are belong to one user or not.
\end{itemize}

\subsection{Proof of Security}
\noindent\textbf{Authentication.}

The  confidentiality of identity token is guaranteed because none of the honest would send it to a malicious party. The detail of the proof is shown in the Appendix. Here we only focus on the main point of confidentiality, preventing the honest parties from sending the identity token to the adversary without $redirect_uri$ mechanism. IdP issues the signed $RP certificate$ for each RP, which contains the RP's correct endpoint for identity token. It can be found in Figure \ref{fig:process}, in step 4.5-4.6, the identity token is generated and sent ton RP, and the identity token is strictly transformed to the origin defined in $RP certificate$. The scheme achieves the same security property as $redirect_uri$ mechanism.

The integrity of identity token is guaranteed as all the attributes contained are well protected by the signature, and the key is never leaked to the adversary. Moreover, it can be proved that the attributes included in the identity token cannot be controlled by the adversary beside of $PID_{RP}$.

Due to the $PID_{RP}$ registration, in a valid period an $PID_{RP}$ is only available to an RP, restricted by the $hash(N_U)$. Therefore, in the valid period the identity token can be only accepted by the specific RP.

The main point of identification is whether an RP accepted $Account$ can be controlled by an adversary may be the most noticeable question for readers.
For example, the adversary may try to make the conflict $Account_1=ID_{U_1}ID_{RP_1}$, $Account_2=ID_{U_2}ID_{RP_2}$, and $Account_1=Account_2$ possible, where $ID_{U_1}$ and $ID_{RP_1}$ belong to the honest user and RP.
Here we give the direct conclusion. \textbf{The $PID_U$ achieved by an adversary cannot be transformed into the honest user's $Account$ at an honest RP.}

The details of authentication proof are shown in Appendix.


\subsection{Proof of Privacy}

In this section, we will give the privacy proof and show that UPPRESSO is secure against both IdP-based login tracing and RP-based identity linkage attacks.

\noindent\textbf{IdP-based login tracing.}
As shown in figure \ref{fig:process}, the only information that is related to the RP's identity and is accessible to the IdP is $PID_{RP}$, which is converted from $ID_{RP}$ using a random $N_U$. Since $N_U$ is randomly chosen from $\mathbb{Z}_n$ by the user and the IdP ha no control of the process, the IdP should treat $PID_{RP}$ as being randomly chosen from $\mathbb{G}$. So, the IdP cannot recognize the RP nor derive its real identity. Therefore, IdP-based identity linkage becomes impossible in UPPRESSO.



Here, we give the conclusion that, \textbf{while adversaries can distinguish whether multiple login flows targeting different RPs belong to the same user or not, they are able to solve Decisional Diffie-Hellman (DDH) problem \cite{GoldwasserK16}.}  The demonstration is provided in this section, and the details of formal analysis is provided in Appendix \ref{}.
Following, we briefly introduce the DDH assumption.

\noindent\textbf{The DDH Assumption.}
Let $q$ be a large prime and $\mathbb{G}$ denotes a cyclic group of order $n$ of an elliptic curve $E(\mathbb{F}_q)$.
Assume that $n$ is also a large prime. Let $P$ be a generator point of $\mathbb{G}$. The DDH assumption for $\mathbb{G}$ states that for any probabilistic polynomial time (PPT) algorithm $D$, the two probability distributions \{$aP$, $bP$, $abP$\} and \{$aP$, $bP$, $cP$\}, where $a$, $b$, and $c$ are randomly and independently chosen from $\mathbb{Z}_n$, are computationally indistinguishable in the sense that there is a negligible $\sigma(n)$ with the security parameter $n$ such that:
\vspace{-\topsep}
\begin{multline*}
Pr[D(P, aP, bP, abP)=1]-Pr[D(P, aP, bP, cP)=1]=\sigma(n)
\end{multline*}
\vspace{-\topsep}


Now, we assume that there is the probabilistic polynomial time (PPT) algorithm $D$, having advantage on guessing whether two login flows targeting different RPs belong to the same user. That is, $D$ is considered as the black box, while the input is the data received by RPs in two login flows. And the output is 1 while the logins flows belong to the same user, otherwise, the output is 0. The advantage means that,   $Pr[D(login\ from\ same\ user)=1]-Pr[D(login\ from\ different\ users)=1]>\sigma(n)$.  The

