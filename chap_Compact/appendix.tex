
\begin{appendices}

\renewcommand{\algorithmicrequire}{\textbf{Input:}}  
\newcommand{\deflet}{\textbf{let}}
\newcommand{\mystate}[1]{\STATE \textbf{let} {{}#1}}
\newcommand{\mystop}[1]{\STATE \textbf{stop} \myss{\myangle{{{}#1}}, s'}}
\newcommand{\myss}[1]{${{}#1}$}
\newcommand{\myangle}[1]{\langle {{}#1} \rangle}
\newcommand{\myif}[1]{\IF{\myss{{{}#1}}}}
\newcommand{\myelse}[1]{\ELSIF{\myss{{{}#1}}}}

\newcommand{\SWITCH}[1]{\STATE \textbf{switch} #1\ \textbf{do} \begin{ALC@g}}
\newcommand{\ENDSWITCH}{\end{ALC@g}\STATE \textbf{end switch}}
\newcommand{\CASE}[1]{\STATE \textbf{case} #1\textbf{:} \begin{ALC@g}}
\newcommand{\ENDCASE}{\end{ALC@g}}
\newcommand{\CASELINE}[1]{\STATE \textbf{case} #1\textbf{:} }
\newcommand{\DEFAULT}{\STATE \textbf{default:} \begin{ALC@g}}
\newcommand{\ENDDEFAULT}{\end{ALC@g}}
\newcommand{\DEFAULTLINE}[1]{\STATE \textbf{default:} }

\section{Web Model}
\subsection{Data Formate}
Here we provide the details of formate of some messages we use to construct the UPPRESSO model.

\vspace{1mm}\noindent\textbf{HTTP Messages}.
An HTTP request message is the term of the form 
\begin{equation*}
\myangle{\mathtt{HTTPReq}, nonce, method, host, path, parameters, headers, body}
\end{equation*}
, and an HTTP response message is the term of the form 
\begin{equation*}
    \myangle{\mathtt{HTTPResp}, nonce, status, headers, body}
\end{equation*}
 The details are dined as follows:
 \begin{itemize}
 \item \myss{\mathtt{HTTPReq}} and \myss{\mathtt{HTTPResp}} are the type of messages.
 \item \myss{nonce} is the constant nonce mapping the response with the specific request.
 \item \myss{method} is the HTTP method, such as \myss{\mathtt{GET}} and \myss{\mathtt{POST}}.
 \item \myss{host} is the constant string domain of visited server.
 \item \myss{path} is the constant string representing the concrete resource of the server.
 \item \myss{parameters} contains the parameters carried by the url as the form \myss{\myangle{\myangle{name, value}, \myangle{name, value}, \dotsc}}, , for example the \myss{parameters} HTTP request sent to the url \myss{http://www.example.com?type=confirm}  is \myss{\myangle{\myangle{type, confirm}}}.
 \item \myss{headers} is the header content of each HTTP messages as the form \myss{\myangle{\myangle{name, value}, \myangle{name, value}, \dotsc}}, such as \myss{\myangle{\myangle{Referer, http://www.example.com}, \myangle{Cookies, c}}}.
 \item \myss{body} is the body content carried by HTTP \myss{\mathtt{POST}} request or HTTP response in the form \myss{\myangle{\myangle{name, value}, \myangle{name, value}, \dotsc}}.
  \item \myss{status} is the HTTP status code defined by HTTP standard.
 \end{itemize}
 
\vspace{1mm}\noindent\textbf{URL}.
A URL is a term \myss{\myangle{\mathtt{URL}, protocol,host,path,parameters}}, where \myss{\mathtt{URL}} is the type, \myss{protocol} is chosen in {\myss{\mathtt{S}}, \myss{\mathtt{P}}} as \myss{\mathtt{S}} stands for HTTPS and \myss{\mathtt{P}} stands for HTTP. The \myss{host, path, and parameters} are same as in HTTP message. 

\vspace{1mm}\noindent\textbf{Origin}. 
An Origin is a term \myss{\myangle{host, protocol}} that stands for the specific domain used by the HTTP CORS policy, where \myss{host} and \myss{protocol} are defined as same as in URL.

\vspace{1mm}\noindent\textbf{POSTMESSAGE}.
PostMessage is used in the browser for transmitting messages between scripts from different origins. We define the postMessage as the form \myss{\myangle{\mathtt{POSTMESSAGE}, target, Content, Origin}}, where \myss{\mathtt{POSTMESSAGE}} is the type, \myss{target} is the constant nonce which stands the for the receiver, \myss{Content} is the message transmitted and Origin is restricts the receiver's origin.

\vspace{1mm}\noindent\textbf{XMLHTTPREQUEST}. 
XMLHTTPRequest is the HTTP message transmitted  by scripts in the browser. That is the XMLHTTPRequest is converted with the HTTP message by the browser. The XMLHTTPRequest in the form \myss{\myangle{\mathtt{XMLHTTPREQUEST}, URL, methods, Body, nonce}} can be converted into HTTP request message by the browser,and \myss{\myangle{\mathtt{XMLHTTPREQUEST}, Body, nonce}}  is converted from HTTP response message.

\vspace{1mm}\noindent\textbf{Data Operation}. 
The data used in UPPRESSO are defined in the following forms:
\begin{itemize}
\item \textbf{Standardized Data} is the data in the fixed format, for instance the HTTP request is the standardized data in the form \myss{\myangle{\mathtt{HTTPReq}, nonce, method, host, path, parameters, headers, body}}.  We assume there is an HTTP request \myss{r := \myangle{\mathtt{HTTPReq}, n, \mathtt{GET}, example.com, /path, \myangle{}, \myangle{}, \myangle{}}}, here we define the operation on the $r$. That is the elements in $r$ can be accessed in the form \myss{r.name}, such that \myss{r.method \equiv \mathtt{GET}},  \myss{r.path \equiv /path} and \myss{r.body \equiv \myangle{}}. 
\item \textbf{Dictionary Data} is the data in the form \myss{\myangle{\myangle{name, value}, \myangle{name, value}, \dotsc}}, for instance the \myss{body} in HTTP request is dictionary data. We assume there is a \myss{body := \myangle{\myangle{username, alice}, \myangle{password, 123}}}, here we define the operation on the $body$. That is we can access the elements in \myss{body} in the form \myss{body[name]}, such that \myss{body[username] \equiv alice} and \myss{body[password] \equiv 123}. We can also add the new attributes to the dictionary, for example after we set \myss{body[age] := 18}, the \myss{body} are changed into\myss{ \myangle{\myangle{username, alice}, \myangle{password, 123}, \myangle{age, 18}}}.
\end{itemize}

\vspace{1mm}\noindent\textbf{Patten Matching}.  We define the term with the variable $\star$ as the pattern, such as \myss{\myangle{a, b, \star}}. That is the pattern matches any terms which is in the same form of the pattern but replacing the $\star$ with other terms. For instance,  \myss{\myangle{a, b, \star}} matches \myss{\myangle{a, b, c}}.

\subsection{Browser Model}
As we consider that  the browsers are honest in UPPRESSO model, therefore, we only focus on how the browsers interactive with the scripts. 

We firstly introduce the windows and documents of the browser model.

\vspace{1mm}\noindent\textbf{Window}. A window \myss{w} is a term of the form \myss{w = \myangle{nonce, documents, opener}}, representing the  the concrete browser window in the system. The \myss{nonce} is the window reference to identify each windows. The \myss{documents} is the set of documents (defined below) including the current document and cached documents (for example, the documents can be viewed via the "forward" and "back" buttons in the browser). The \myss{opener} represents the widow is created in which document, for instance, while a user clicks the href in document \myss{d} and it creates a new window \myss{w}, there is \myss{w.opener \equiv d.nonce}. 

\vspace{1mm}\noindent\textbf{Document}. A document \myss{d} is a term of the form
\begin{equation*}
  \myangle{nonce, location, referrer, script, scriptstate, scriptinputs, subwindows, active}
\end{equation*} 
where document is the HTML content in the window.  The \myss{nonce} is to locate the document. \myss{Location} is the URL where the document is loaded. \myss{Referrer} is same as the Referer header defined in HTTP standard. \myss{script} is the scripting process downloaded from each servers. \myss{scriptstate} is define by the script, different in each scripts.  \myss{scriptinputs} is the message transmitted into the scripting process. \myss{subwindows} is the set of \myss{nonce} of document's created windows. \myss{active} represents whether this document is active or not.

A scripting process is the dependent process relying on the browser, which can be considered as a relation \myss{R} mapping a message input and a message output. And finally the browser will conduct the command in the output message. Here we give the description of the form of input and output. 
\begin{itemize}
\item \textbf{Scripting Message Input. } The input is the term in the form
\begin{equation*}
\myangle{tree, docnonce, scriptstate, stateinputs,cookies,localStorage, sessionStorage, ids, secret}
\end{equation*}
\item \textbf{Scripting Message Output. }The output is the term in the form
\begin{equation*}
\myangle{scriptstate, cookies, localStorage, sessionStorage, command}
\end{equation*}
\end{itemize}
The \myss{tree} is the relations of the opened windows and documents, which are visible to this script. \myss{Docnonce} is the document nonce. The  \myss{Scriptstate} is a term of the form defined by each script. \myss{Scriptinputs} is the message transmitted to script. However, the \myss{scriptinputs} are defined as standardized forms, for example postMessage is one of the forms of \myss{scriptinputs}. \myss{Cookies} is the set of cookies belong to the document's origin. \myss{LocalStorage} is the storage space for browser and \myss{sessionStorage} is the space for each HTTP sessions.  \myss{Ids} is the set of user IDs while \myss{secret} is the password to corresponding user ID. The \myss{command} is the operation which is to be conducted by the browser. Here we only introduce the form of commands used in UPPRESSO system. We have defined the postMessage and XMLHTTPRequest (for HTTP request) message which are the \myss{commands}. Moreover, a term in the form \myss{\myangle{\mathtt{IFRAME}, URL, WindowNonce}} asks the browser to create this document's subwindow and it visits the server with the URL.



\subsection{Model Of UPPRESSO}
In this section, we will introduce the model of processes in UPPRESSO system, containing IdP server process, RP server process, IdP scripting process and RP scripting process. We will focus on the state form and relation $R$. They can describe that what kind of event can be accepted by the process in each states, and the content of new output events and states.
\subsection{IdP Server Process}
The state of IdP server process is a term in the form \myss{\myangle{ID, p, SignKey, sessions, users, RPs, Validity, Tokens}}. Other data stored at IdP but not used during SSO authentication are not mentioned here.
\begin{itemize}
\item \myss{ID} is the identifier of IdP.
\item \myss{p} is the large prime mentioned before.
\item \myss{SignKey} is the private key used by IdP to generate signatures.
\item \myss{sessions} is the term in the form of \myss{\myangle{\myangle{Cookie, session}}}, the Cookie uniquely identify the session and sessions store the  browser uploaded message.
\item \myss{users} is the set of user information. And each user informations contains the \myss{username, password, uid}  and other user attributes.
\item \myss{RPs} is the set of RP information which consists of ID of RP \myss{(PID_{RP})}, \myss{Endpoints} the set of RP's validity endpoints and \myss{Validity}. 
\item \myss{Validity} is the validity for IdP generated signatures. 
\item \myss{Tokens} is the set of IdP generated Identity proofs.
\end{itemize}
To make the description clearer, we also provide the \myss{functions} to define the complicated procedure. 
\begin{itemize}
\item \myss{\mathtt{SecretOfID(u)}} is used to search the user \myss{u}'s password.
\item \myss{\mathtt{UIDOfUser(u)}} is used to search the user \myss{u}'s \myss{uid}.
\item \myss{\mathtt{ListOfPID()}} is the set of IDs of registered RP.
\item \myss{\mathtt{EndpointsOfRP(r)}} is the set of endpoints registered by the RP with ID \myss{r}. 
\item \myss{\mathtt{ModPow(a, b, c)}} is the result of \myss{a^b \mod c}.
\item \myss{\mathtt{CurrentTime()}} is the system current time.
\end{itemize}

The relation of IdP process $R^i$ is shown as Algorithm~\ref{alg1}.

\begin{breakablealgorithm}
  \caption{$R^i$}
  \label{alg1}
  \begin{algorithmic}[1]
  \REQUIRE \myss{\myangle{a, f, m}, s}
  \mystate{\myss{s:=s'}}
  \mystate{\myss{n, method, path, parameters, headers, body} \textbf{such that}}\\
  \ \ \myss{\myangle{\mathtt{HTTPReq},n,method,path,parameters,headers,body} \equiv m}\\
  \ \ \textbf{if} \myss{possible}; \textbf{otherwise} stop \myss{\myangle{}, s'}
  \myif{path \equiv /script}
  \mystate{\myss{m':=\myangle{\mathtt{HTTPResp},n,200, \myangle{}, \mathtt{IdPScript}}}}
  \mystop{f, a, m'}
  \myelse{path \equiv /login}   
  \mystate{\myss{cookie := headers[Cookie]}}
  \mystate{\myss{session := s'.sessions[cookie]}}
  \mystate{\myss{username:=body[username]}}
  \mystate{\myss{password:=body[password]}}
  \myif{password \not\equiv \mathtt{SecretOfID}(username)}
  \mystate{\myss{m' :=\myangle{\mathtt{HTTPResp},n,200,\myangle{},\mathtt{LoginFailure}}}}
  \mystop{f,a,m'}
  \ENDIF
  \mystate{\myss{session[uid] := \mathtt{UIDOfUser}(username)}}
  \mystate{\myss{m' :=\myangle{\mathtt{HTTPResp},n,200,\myangle{},\mathtt{LoginSucess}}}}
  \mystop{f,a,m'}
  \myelse{path \equiv /loginInfo}
  \mystate{\myss{cookie := headers[Cookie]}}
  \mystate{\myss{session := s'.sessions[cookie]}}
  \mystate{\myss{username := session[username]}}
  \myif{username \not\equiv \mathtt{null}}
  \mystate{\myss{m' := \myangle{\mathtt{HTTPResp},n,200,\myangle{},\mathtt{Logged}}}}
  \mystop{f,a,m'}
  \ENDIF
  \mystate{\myss{m' := \myangle{\mathtt{HTTPResp},n,200,\myangle{},\mathtt{Unlogged}}}}
  \mystop{f,a,m'}
  \myelse{path \equiv /dynamicRegistration}
  \mystate{\myss{PID_{RP} := body[PID_{RP}]}}
 \mystate{\myss{Endpoint := body[Endpoint]}}
  \mystate{\myss{Nonce := body[Nonce]}}
  \myif{PID_{R}P \in \mathtt{ListOfPID()}}
  \mystate{\myss{Content :=\myangle{Fail, PID_{RP}, Nonce}}}
  \mystate{\myss{Sig := \mathtt{Sig}(Content, s'.SignKey)}}
  \mystate{\myss{RegistrationResult := \myangle{Content, Sig}}}
  \mystate{\myss{m' := \myangle{\mathtt{HTTPResp}, n, 200, \myangle{}, RegistrationResult}}}
  \mystop{f,a,m'}
  \ENDIF
  \mystate{\myss{Validity := \mathtt{CurrentTime} ()+ s'.Validity}}
 \mystate{\myss{s'.RPs := s'.RPs +  ^{\myangle{}} \myangle{PID_{RP}, Endpoint, Validity}}}
  \mystate{\myss{Content := \myangle{OK, PID_{RP}, Nonce, Validity}}}
  \mystate{\myss{Sig := \mathtt{Sig}(Content, s'.SignKey)}}
  \mystate{\myss{RegistrationResult := \myangle{Content, Sig}}}
  \mystate{\myss{m' := \myangle{\mathtt{HTTPResp}, n, 200, \myangle{}, RegistrationResult}}}
  \mystop{f,a,m'}
  \myelse{path \equiv /authorize}
  \mystate{\myss{cookie := headers[Cookie]}}
  \mystate{\myss{session := s'.sessions[cookie]}}
  \mystate{\myss{username := session[username]}}
  \myif{username \equiv \mathtt{null}}
  \mystate{\myss{m' := \myangle{\mathtt{HTTPResp}, n, 200, \myangle{}, \mathtt{Fail}}}}
  \mystop{f,a,m'}
  \ENDIF
  \mystate{\myss{PID_{RP} := parameters[PID_{RP}]}}
  \mystate{\myss{Endpoint := parameters[Endpoint]}}
  \myif{PID_{RP} \notin \mathtt{ListOfPID}() \lor Endpoint \notin \mathtt{EndpointsOfRP}(PID)}
  \mystate{\myss{m' := \myangle{\mathtt{HTTPResp}, n, 200, \myangle{}, \mathtt{Fail}}}}
  \mystop{f,a,m'}
  \ENDIF
  \mystate{\myss{UID := session[uid]}}
  \mystate{\myss{PID_U := \mathtt{ModPow}(PID_{RP}, UID, s'.p)}}
  \mystate{\myss{Validity := \mathtt{CurrentTime} ()+ s'.Validity}}
  \mystate{\myss{Content := \myangle{PID_{RP}, PID_U, s'.ID, Validity}}}
  \mystate{\myss{Sig := \mathtt{Sig}(Content, s'.SignKey)}}
  \mystate{\myss{Token := \myangle{Content, Sig}}}
  \mystate{\myss{s'.Tokens := s'.Tokens + ^{\myangle{}}Token}}
  \mystate{\myss{m' := \myangle{\mathtt{HTTPResp}, n, 200, \myangle{}, \myangle{Token, Token}}}}
  \mystop{f, a, m'}
  \ENDIF
  \mystop{}
  \end{algorithmic}
\end{breakablealgorithm}


\subsection{RP process}
The state of RP server process is a term in the form \myss{\myangle{ID_{RP}, Endpoints, IdP, Cert, sessions, users}}. Other attributes are not mentioned here. 
\begin{itemize}
\item \myss{ID_{RP}} and \myss{Endpoints} are RP's registered information at IdP.
\item \myss{Cert} is the IdP signed RP information containing \myss{ID_{RP}, Endpoints} and other attributes.
\item \myss{IdP} is the term of the for \myss{\myangle{ScriptUrl, p, q, PubKey}}, where \myss{ScriptUrl} is the site to download IdP script, \myss{p} and \myss{q} are large prime defined before, and \myss{PubKey} is used to verify the IdP signed content.
\item \myss{sessions} is same as it in RP process.
\item \myss{users} is the set of RP registered user which is uniquely identified by the \myss{Account}.
\end{itemize}
 The new \myss{functions} are defined as follows
 \begin{itemize}
 \item \myss{\mathtt{ExEU}(a, q)} is the Extended Euclidean algorithm, of which the result in RP process the \myss{a^{-1} \mod q}.
  \item \myss{\mathtt{Random}()} is a newly generated random number.
  \item \myss{\mathtt{RegisterUser}(Account)} add the new user with \myss{Account} into RP's user list.
 \end{itemize}
The relation of RP process $R^r$ is shown as Algorithm~\ref{alg2}.

\begin{breakablealgorithm}
  \caption{$R^r$}
  \label{alg2}
  \begin{algorithmic}[1]
  \REQUIRE \myss{\myangle{a, f, m}, s}
  \mystate{\myss{s:=s'}}
  \mystate{\myss{n, method, path, parameters, headers, body} \textbf{such that}}\\
  \ \ \myss{\myangle{\mathtt{HTTPReq},n,method,path,parameters,headers,body} \equiv m}\\
  \ \ \textbf{if} \myss{possible}; \textbf{otherwise} stop \myss{\myangle{}, s'}
  \myif{path \equiv /script}
\mystate{\myss{m':=\myangle{\mathtt{HTTPResp},n,200, \myangle{}, \mathtt{RPScript}}}}
  \mystop{f, a, m'}  
  \myelse{path \equiv /login}   
  \mystate{\myss{m'  := \myangle{\mathtt{HTTPResp},n,302,\myangle{\myangle{Location, s'.IdP.ScriptUrl}}, \myangle{}}}}
  \mystop{f, a, m'}
  \myelse{path \equiv /startNegotiation}
  \mystate{\myss{cookie := headers[Cookie]}}
  \mystate{\myss{session := s'.sessions[cookie]}}
  \mystate{\myss{N_U := parameters[N_U]}}
  \mystate{\myss{PID_{RP} := \mathtt{ModPow}(s'.ID_{RP}, N_U, s'.IdP.p)}}
  \mystate{\myss{t := \mathtt{ExEU}(N_U, s'.IdP.q)}}
  \mystate{\myss{session[N_U] := N_U}}
  \mystate{\myss{session[PID_{RP}] := PID_{RP}}}
  \mystate{\myss{session[t] := t}}
  \mystate{\myss{session[state] := expectRegistration}}
  \mystate{\myss{m' := \myangle{\mathtt{HTTPResp}, n, 200, \myangle{}, \myangle{Cert, s'.Cert}}}}
 \mystop{f, a, m'}
 \myelse{path \equiv /registrationResult}
 \mystate{\myss{cookie := headers[Cookie]}}
  \mystate{\myss{session := s'.sessions[cookie]}}
  \myif{session[state] \not\equiv expectRegistration}
  \mystate{\myss{m' := \myangle{\mathtt{HTTPResp}, n, 200, \myangle{}, \mathtt{Fail}}}}
  \mystop{f, a, m'}
  \ENDIF
  \mystate{\myss{RegistrationResult := body[RegistrationResult]}}
  \mystate{\myss{Content:=RegistrationResult.Content}}
  \myif{\mathtt{checksig}(Content, RegistrationResult.Sig, s'.IdP.PubKey) \equiv \mathtt{FALSE}}
  \mystate{\myss{m' := \myangle{\mathtt{HTTPResp}, n, 200, \myangle{}, \mathtt{Fail}}}}
  \mystate{\myss{session := \mathtt{null}}}
  \mystop{f, a, m'}
  \ENDIF
  \myif{Content.Result \not\equiv OK}
  \mystate{\myss{m' := \myangle{\mathtt{HTTPResp}, n, 200, \myangle{}, \mathtt{Fail}}}}
  \mystate{\myss{session := \mathtt{null}}}
  \mystop{f, a, m'}
  \ENDIF
  \mystate{\myss{PID_{RP} := session[PID_{RP}]}}
  \mystate{\myss{N_U := session[N_U]}}
  \mystate{\myss{Nonce := \mathtt{Hash}( N_U)}}
  \mystate{\myss{Time := \mathtt{CurrentTime}()}}
  \myif{PID_{RP} \not\equiv Content.PID_{RP} \lor Nonce \not\equiv Content.Nonce \lor Time > Content.Validity}
  \mystate{\myss{m' := \myangle{\mathtt{HTTPResp}, n, 200, \myangle{}, \mathtt{Fail}}}}
  \mystate{\myss{session := \mathtt{null}}}
  \mystop{f, a, m'}
  \ENDIF
  \mystate{\myss{session[PIDValidity] := Content.Validity}}
  \mystate{\myss{Endpoint \in s'.Endpoints}}
  \mystate{\myss{session[state] := expectToken}}
  \mystate{\myss{Nonce' := \mathtt{Random}()}}
  \mystate{\myss{session[Nonce] := Nonce'}}
  \mystate{\myss{Body := \myangle{PID_{RP}, Endpoint, Nonce'}}}
  \mystate{\myss{m' := \myangle{\mathtt{HTTPResp}, n, 200, \myangle{}, Body}}}
  \mystop{f, a, m'}
  \myelse{path \equiv /uploadToken}
 \mystate{\myss{cookie := headers[Cookie]}}
  \mystate{\myss{session := s'.sessions[cookie]}}
  \myif{session[state] \not\equiv expectToken}
  \mystate{\myss{m' := \myangle{\mathtt{HTTPResp}, n, 200, \myangle{}, \mathtt{Fail}}}}
  \mystop{f, a, m'}
  \ENDIF
  \mystate{\myss{Token := body[Token]}}
  \myif{\mathtt{checksig}(Token.Content, Token.Sig, s'.IdP.PubKey) \equiv \mathtt{FALSE}}
  \mystate{\myss{m' := \myangle{\mathtt{HTTPResp}, n, 200, \myangle{}, \mathtt{Fail}}}}
  \mystop{f, a, m'}
  \ENDIF
  \mystate{\myss{PID_{RP} := session[PID_{RP}]}}
  \mystate{\myss{Time := \mathtt{CurrentTime}()}}
  \mystate{\myss{PIDValidity := session[PIDValidity]}}
  \mystate{\myss{Content := Token.Content}}
  \myif{PID_{RP} \not\equiv Content.PID_{RP} \lor Time>Content.Validity \lor Time>PIDValidity}
  \mystate{\myss{m' := \myangle{\mathtt{HTTPResp}, n, 200, \myangle{}, \mathtt{Fail}}}}
  \mystop{f, a, m'}
  \ENDIF
  \mystate{\myss{PID_U := Content.PID_U}}
  \mystate{\myss{t := session[t]}}
  \mystate{\myss{Account := \mathtt{ModPow}(PID_U, t, s'.IdP.p)}}
  \myif{Account \in \mathtt{ListOfUser}()}
  \mystate{\myss{\mathtt{RegisterUser}(Account)}}
  \ENDIF
  \mystate{\myss{session[user] := Account}}
  \mystate{\myss{m' := \myangle{\mathtt{HTTPResp}, n, 200, \myangle{}, \mathtt{LoginSuccess}}}}
  \mystop{f, a, m'}
  \ENDIF
  \mystop{}
  \end{algorithmic}
\end{breakablealgorithm}

\subsection{IdP scripting process}
The state of IdP scripting process \myss{scriptstate} is a term in the form \myss{\myangle{IdPDomain, Parameters,p, q, refXHR}}, where 
\begin{itemize}
\item \myss{IdPDomain} is the IdP's host.
\item \myss{Parameters} is used to store the parameters received from other process.
\item \myss{p} is the large prime defined before.
\item \myss{q} is used to label the procedure point in the login.
\item \myss{refXHR} is the nonce to mapping HTTP request and response.
\end{itemize}
 The new \myss{functions} are defined as follows
 \begin{itemize}
 \item \myss{\mathtt{PARENTWINDOW}(tree,docnonce)}. The first parameter is the input relation tree defined before, and the second parameter is the nonce of a document. The output returned by the function is the current window's opener's nonce if it exists and is visible to this document.  
  \item \myss{\mathtt{CHOOSEINPUT}(inputs,pattern)}. The first parameter is a set of messages, and the second parameter is a pattern. The result returned by the function is the message in \myss{inputs} matching the \myss{pattern}.
  \item \myss{\mathtt{RandonUrl}()} returns a newly generated host string. 
 \end{itemize}
 The relation of IdP scripting process $script\_idp$ is shown as Algorithm~\ref{alg3}.

\begin{breakablealgorithm}
  \caption{$script\_idp$}
  \label{alg3}
  \begin{algorithmic}[1]
  \REQUIRE \myss{\myangle{tree, docnonce, scriptstate, scriptinputs, cookies, localStorage, sessionStorage, ids, secret}}
  \mystate{\myss{ s' := scriptstate}}
  \mystate{\myss{command := \myangle{}}}
  \mystate{\myss{target := \mathtt{PARENTWINDOW}(tree,docnonce)}}
  \mystate{\myss{IdPDomain := s'.IdPDomain}}
  \SWITCH{\myss{s'.q}}
    \CASE{\myss{start}}
      \mystate{\myss{N_U := \mathtt{Random}()}}
      \mystate{\myss{command := \myangle{\mathtt{POSTMESSAGE}, target, \myangle{\myangle{N_U, N_U}}, \mathtt{null}}}}
      \mystate{\myss{s'.Parameters[N_U] := N_U}}
      \mystate{\myss{s'.q := expectCert}}
    \ENDCASE
    \CASE{\myss{expectCert}}
      \mystate{\myss{pattern := \myangle{\mathtt{POSTMESSAGE}, *, Content, *}}}
      \mystate{\myss{input := \mathtt{CHOOSEINPUT}(scriptinputs,pattern)}}
      \myif{input \not\equiv \mathtt{null}}
      \mystate{\myss{Cert := input.Content[Cert]}}
      \myif{\mathtt{checksig}(Cert.Content, Cert.Sig, s'.PubKey) \equiv \mathtt{null}}
      \mystate{\myss{\textbf{stop}\ \myangle{}}}
      \ENDIF
       \mystate{\myss{s'.Parameters[Cert] := Cert}}
      \mystate{\myss{N_U := s'.Parameters[N_U]}}
      \mystate{\myss{PID_{RP} := \mathtt{ModPow}(Cert.Content.ID_{RP}, N_U, s'.p)}}
      \mystate{\myss{s'.Parameters[PID_{RP}] := PID_{RP}}}
      \mystate{\myss{Endpoint := \mathtt{RandomUrl}()}}
      \mystate{\myss{s'.Parameters[Endpoint] := Endpoint}}
      \mystate{\myss{Nonce := \mathtt{Hash}(N_U)}}
      \mystate{\myss{Url := \myangle{\mathtt{URL}, \mathtt{S}, IdPDomain, /dynamicRegistration,\myangle{} }}}
      \mystate{\myss{s'.refXHR :=  \mathtt{Random}()}}
      \mystate{\myss{command : = \langle\mathtt{XMLHTTPREQUEST}, Url, \mathtt{POST},} \\\ \ \ \ \myss{ \myangle{\myangle{PID_{RP}, PID_{RP}}, \myangle{Nonce, Nonce}, \myangle{Endpoint, Endpoint}}, s'.refXHR\rangle}}
      \mystate{\myss{s'.q := expectRegistrationResult}}
       \ENDIF
      \ENDCASE
      \CASE{expectRegistrationResult}
      \mystate{\myss{pattern := \myangle{\mathtt{XMLHTTPREQUEST},Body,s'.refXHR}}}
      \mystate{\myss{input := \mathtt{CHOOSEINPUT}(scriptinputs,pattern) }}
      \myif{input \not\equiv \mathtt{null} \land input.Content[RegistrationResult].type \equiv OK}
      \mystate{\myss{RegistrationResult := input.Body[RegistrationResult]}}
      \myif{ RegistrationResult.Content.Result \not\equiv OK}
      \mystate{\myss{s'.q := stop}}
      \mystate{\myss{\textbf{stop}\ \myangle{}}}
      \ENDIF
      \mystate{\myss{command := \myangle{\mathtt{POSTMESSAGE}, target, \myangle{\myangle{RegistrationResult, RegistrationResult}}, \mathtt{null}}}}
      \mystate{\myss{s'.q := expectProofRquest}}
      \ENDIF
      \ENDCASE
      \CASE{expectProofRquest}
      \mystate{\myss{pattern := \myangle{\mathtt{POSTMESSAGE}, *, Content, *}}}
      \mystate{\myss{input := \mathtt{CHOOSEINPUT}(scriptinputs,pattern)}}
      \myif{input \not\equiv \mathtt{null}}
       \mystate{\myss{PID_{RP} := input.Content[PID_{RP}]}}
       \mystate{\myss{Endpoint_{RP} := input.Content[Endpoint]}}
       \mystate{\myss{s'.Parameters[Nonce] := input.Content[Nonce]}}
       \mystate{\myss{Cert := s'.Parameters[Cert]}}
      \myif{Endpoint_{RP} \notin Cert.Content.Endpoints \lor PID_{RP} \not\equiv s'.Parameters[PID_{RP}]}
      \mystate{\myss{s'.q := stop}}
      \mystate{\myss{\textbf{stop}\ \myangle{}}}
      \ENDIF
       \mystate{\myss{s'.Parameters[Endpoint_{RP}] := Endpoint_{RP}}}
      \mystate{\myss{Url := \myangle{\mathtt{URL}, \mathtt{S}, IdPDomain, /loginInfo, \myangle{}}}}
      \mystate{\myss{s'.refXHR :=  \mathtt{Random}()}}
      \mystate{\myss{command : = \myangle{\mathtt{XMLHTTPREQUEST}, Url, \mathtt{GET},\myangle{}, s'.refXHR}}}
      \mystate{\myss{s'.q := expectLoginState}}
      \ENDIF
      \ENDCASE
      \CASE{expectLoginState}      
      \mystate{\myss{pattern := \myangle{\mathtt{XMLHTTPREQUEST},Body,s'.refXHR}}}
      \mystate{\myss{input := \mathtt{CHOOSEINPUT}(scriptinputs,pattern) }}
      \myif{input \not\equiv \mathtt{null}}
      \myif{input.Body \equiv \mathtt{Logged}}
      \mystate{\myss{username \in ids}}
      \mystate{\myss{Url := \myangle{\mathtt{URL}, \mathtt{S}, IdPDomain, /login, \myangle{}}}}
     \ mystate{\myss{s'.refXHR :=  \mathtt{Random}()}}
      \mystate{\myss{command : = \myangle{\mathtt{XMLHTTPREQUEST}, Url, \mathtt{POST},\myangle{\myangle{username, username}, \myangle{password, secret}}, s'.refXHR}}}
      \mystate{\myss{s'.q := expectLoginResult}}
      \myelse{input.Body \equiv \mathtt{Unlogged}}
      \mystate{\myss{PID_{RP} := s'.Parameters[PID_{RP}]}}
      \mystate{\myss{Endpoint := s'.Parameters[Endpoint]}}
      \mystate{\myss{Nonce := s'.Parameters[Nonce]}}
      \mystate{\myss{Url := \langle \mathtt{URL}, \mathtt{S}, IdPDomain, /authorize,}\\\ \ \ \  \myss{\myangle{\myangle{PID_{RP}, PID_{RP}}, \myangle{Endpoint, Endpoint}, \myangle{Nonce, Nonce}} \rangle}}
      \mystate{\myss{s'.refXHR :=  \mathtt{Random}()}}
      \mystate{\myss{command : = \myangle{\mathtt{XMLHTTPREQUEST}, Url, \mathtt{GET},\myangle{}, s'.refXHR}}}
      \mystate{\myss{s'.q := expectToken}}
      \ENDIF
      \ENDIF
      \ENDCASE
      \CASE{expectLoginResult}
      \mystate{\myss{pattern := \myangle{\mathtt{XMLHTTPREQUEST},Body,s'.refXHR}}}
      \mystate{\myss{input := \mathtt{CHOOSEINPUT}(scriptinputs,pattern) }}
      \myif{input \not\equiv \mathtt{null}}
      \myif{input.Body \not\equiv \mathtt{LoginSuccess}}
      \mystate{\myss{\textbf{stop}\ \myangle{}}}
      \ENDIF
      \mystate{\myss{PID_{RP} := s'.Parameters[PID_{RP}]}}
      \mystate{\myss{Endpoint := s'.Parameters[Endpoint]}}
      \mystate{\myss{Nonce := s'.Parameters[Nonce]}}
      \mystate{\myss{Url := \langle \mathtt{URL}, \mathtt{S}, IdPDomain, /authorize,}\\\ \ \ \  \myss{\myangle{\myangle{PID_{RP}, PID_{RP}}, \myangle{Endpoint, Endpoint}, \myangle{Nonce, Nonce}} \rangle}}
      \mystate{\myss{s'.refXHR :=  \mathtt{Random}()}}
      \mystate{\myss{command : = \myangle{\mathtt{XMLHTTPREQUEST}, Url, \mathtt{GET},\myangle{}, s'.refXHR}}}
      \mystate{\myss{s'.q := expectToken}}
      \ENDIF
      \ENDCASE
      \CASE{expectToken}
      \mystate{\myss{pattern := \myangle{\mathtt{XMLHTTPREQUEST},Body,s'.refXHR}}}
      \mystate{\myss{input := \mathtt{CHOOSEINPUT}(scriptinputs,pattern) }}
      \myif{input \not\equiv \mathtt{null}}
      \mystate{\myss{Token := input.Body[Token]}}
      \mystate{\myss{RPOringin := \myangle{s'.Parameters[Endpoint_{RP}], \mathtt{S}}}}
      \mystate{\myss{command := \myangle{\mathtt{POSTMESSAGE},target,\myangle{Token,Token},RPOrigin}}}
      \mystate{\myss{s .q := stop}}
     \ENDIF      
    \ENDCASE
  \ENDSWITCH
\mystate{\myss{\textbf{stop}\ \myangle{s',cookies,localStorage,sessionStorage,command}}}
    \end{algorithmic}
\end{breakablealgorithm}

\subsection{RP scripting process}
The state of RP scripting process \myss{scriptstate} is a term in the form \myss{\myangle{IdPDomain, RPDomain, Parameters, q, refXHR}}. The \myss{RPDomain} is the host string of the corresponding RP server, and other terms are defined same as them in IdP scripting process.

Here we define another new function \myss{\mathtt{SUBWINDOW}(tree, docnonce)}. This function takes the \myss{tree} define above and the current document's \myss{nonce} as the input. And it selects the \myss{nonce} of the first window opened by this document as the output. However, if there is not the opened windows, it will return the null.

The relation of RP scripting process $script\_rp$ is shown as Algorithm~\ref{alg4}.
\begin{breakablealgorithm}
  \caption{$script\_rp$}
  \label{alg4}
  \begin{algorithmic}[1]
\REQUIRE \myss{\myangle{tree, docnonce, scriptstate, scriptinputs, cookies, localStorage, sessionStorage, ids, secret}}
\mystate{\myss{ s' := scriptstate}}
  \mystate{\myss{command := \myangle{}}}
  \mystate{\myss{IdPWindow := \mathtt{SUBWINDOW}(tree,docnonce).nonce}}
  \mystate{\myss{RPDomain := s'.RPDomain}}
  \mystate{\myss{IdPOringin := \myangle{s'.IdPDomian, \mathtt{S}}}}
  \SWITCH{\myss{s'.q}}
    \CASE{\myss{start}}
    \mystate{\myss{Url := \myangle{\mathtt{URL}, \mathtt{S}, RPDomain, /login, \myangle{}}}}
    \mystate{\myss{command := \myangle{\mathtt{IFRAME}, Url, \_SELF}}}
    \mystate{\myss{s'.q := expectN_U}}
    \ENDCASE
    \CASE{\myss{expectN_U}}
    \mystate{\myss{pattern := \myangle{\mathtt{POSTMESSAGE}, *, Content, *}}}
      \mystate{\myss{input := \mathtt{CHOOSEINPUT}(scriptinputs,pattern)}}
      \myif{input \not\equiv \mathtt{null}}
      \mystate{\myss{N_U := input.Content[N_U]}}
      \mystate{\myss{Url := \myangle{\mathtt{URL}, \mathtt{S}, RPDomain, /startNegotiation, \myangle{}}}}
      \mystate{\myss{s'.refXHR :=  \mathtt{Random}()}}
      \mystate{\myss{command : = \myangle{\mathtt{XMLHTTPREQUEST}, Url, \mathtt{POST},\myangle{\myangle{N_U, N_U}}, s'.refXHR}}}
      \mystate{\myss{s'.q := expectCert}}
      \ENDIF
      \ENDCASE
      \CASE{\myss{expectCert}}
      \mystate{\myss{pattern := \myangle{\mathtt{XMLHTTPREQUEST},Body,s'.refXHR}}}
      \mystate{\myss{input := \mathtt{CHOOSEINPUT}(scriptinputs,pattern) }}
      \myif{input \not\equiv \mathtt{null}}
      \mystate{\myss{Cert := input.Content[Cert]}}
      \mystate{\myss{command := \myangle{\mathtt{POSTMESSAGE}, IdPWindow, \myangle{\myangle{Cert, Cert}}, IdPOringin}}}
      \mystate{\myss{s'.q := expectRegistrationResult}}
      \ENDIF
      \ENDCASE
      \CASE{\myss{expectRegistrationResult}}
      \mystate{\myss{pattern := \myangle{\mathtt{POSTMESSAGE}, *, Content, *}}}
      \mystate{\myss{input := \mathtt{CHOOSEINPUT}(scriptinputs,pattern)}}
      \myif{input \not\equiv \mathtt{null}}
      \mystate{\myss{RegistrationResult := input.Content[RegistrationResult]}}
      \mystate{\myss{Url := \myangle{\mathtt{URL}, \mathtt{S}, RPDomain, /registrationResult, \myangle{}}}}
      \mystate{\myss{s'.refXHR :=  \mathtt{Random}()}}
      \mystate{\myss{command : = \myangle{\mathtt{XMLHTTPREQUEST}, Url, \mathtt{POST},\myangle{\myangle{RegistrationResult, RegistrationResult}}, s'.refXHR}}}
      \mystate{\myss{s'.q := expectTokenRequest}}
      \ENDIF
      \ENDCASE
      \CASE{\myss{expectTokenRequest}}
      \mystate{\myss{pattern := \myangle{\mathtt{XMLHTTPREQUEST},Body,s'.refXHR}}}
      \mystate{\myss{input := \mathtt{CHOOSEINPUT}(scriptinputs,pattern) }}
      \myif{input \not\equiv \mathtt{null}}
      \mystate{\myss{PID_{RP} := input.Content.Body[PID_{RP}]}}
      \mystate{\myss{Endpoint := input.Content.Body[Endpoint]}}
      \mystate{\myss{Nonce := input.Content.Body[Nonce]}}
      \mystate{\myss{command := \langle\mathtt{POSTMESSAGE}, IdPWindow}, \\\ \ \ \  \myss{\myangle{\myangle{PID_{RP}, PID_{RP}}, \myangle{Endpoint, Endpoint}, \myangle{Nonce, Nonce}},  IdPOringin\rangle}}
      \mystate{\myss{s'.q := expectToken}}
      \ENDIF
      \ENDCASE
      \CASE{\myss{expectToken}}
      \mystate{\myss{pattern := \myangle{\mathtt{POSTMESSAGE}, *, Content, *}}}
      \mystate{\myss{input := \mathtt{CHOOSEINPUT}(scriptinputs,pattern)}}
      \myif{input \not\equiv \mathtt{null}}
      \mystate{\myss{Token := input.Content[Token]}}
      \mystate{\myss{Url := \myangle{\mathtt{URL}, \mathtt{S}, RPDomain, /uploadToken, \myangle{}}}}
      \mystate{\myss{s'.refXHR :=  \mathtt{Random}()}}
      \mystate{\myss{command : = \myangle{\mathtt{XMLHTTPREQUEST}, Url, \mathtt{POST},\myangle{\myangle{Token, Token}}, s'.refXHR}}}
      \mystate{\myss{s'.q := expectLoginResult}}
      \ENDIF
      \ENDCASE
      \CASE{\myss{expectLoginResult}}
      \mystate{\myss{pattern := \myangle{\mathtt{XMLHTTPREQUEST},Body,s'.refXHR}}}
      \mystate{\myss{input := \mathtt{CHOOSEINPUT}(scriptinputs,pattern) }}
      \myif{input \not\equiv \mathtt{null}}
      \myif{input.Body \equiv \mathtt{LoginSuccess}}
      \mystate{\myss{LoadHomepage}}
      \ENDIF
      \ENDIF
    \ENDCASE
    \ENDSWITCH

\end{algorithmic}
\end{breakablealgorithm}

\subsection{Proof of Theorem~\ref{the:secure}}


\newtheorem{redef}{Definition}
\newtheorem{req}{Requirement}
\newtheorem{relemma}{Lemma}

We assume that all the network messages are protected by HTTPS and postMessage messages are protected by the browser, such that web attackers listening the network flow are not considered. 

In this section, we will give a full version of proof about UPPRESSO security. Firstly, we recall the security requirements of UPPRESSO. That is, the system must ensure that only a legitimate user can log into an honest RP under her unique account. We consider the visits to RP's resource paths are controlled by the visitors' cookie, so that the attacker can break the security only when he own the cookie bound to the honest user. Therefore, we can propose the definition~\ref{def:secure} about the secure UPPRESSO system.
\begin{redef}
Let $\mathcal{UWS}$ be a UPPRESSO web system, $\mathcal{UWS}$ is secure \textbf{iff} for any honest RP $r$ $\in $ $\mathcal{W}$ and  the authenticated cookie $c$ for honest $u$,  $c$ is unknown to the attacker $a$. 
\end{redef}
Therefore, the proof of Theorem~\ref{the:secure} is converted into whether the UPPRESSO system meet the requirement in definition~\ref{def:secure}. However, as we consider the attacker initially knows any honest users' cookies, the requirement of definition~\ref{def:secure} can be separated as following requirements. Before describe the requirements, we firstly define the user \myss{u}'s authenticated cookie for RP \myss{r} as \myss{c(u,r)}.
\begin{req}
If $c(u,r)$ is the authenticated cookie owned by $u$, $c(u,r)$ cannot be obtained by $a$.
\label{req:cookie1}
\end{req}
\begin{req}
If $c$ is an unauthenticated cookie owned by $a$, $c$ cannot be set as \myss{c(u,r)}. 
\label{req:cookie2}
\end{req}

To prove that UPPRESSO meets the requirements, we now show some lemmas. The first lemma here can prove that the UPPRESSO system meets requirement~\ref{req:cookie1}.
\begin{relemma}
Attacker does not learn users' cookies.
\label{rel:cookie}
\end{relemma}

\begin{proof}
As the Brute-force attacks, such as exhausting the possible users' cookies, are not  considered, the attackers can only try to obtain the cookies from honest processes in the system.  For an honest user \myss{u} and the honest RP \myss{r}, the valid cookie \myss{c(u,r)} can only be obtained by \myss{u}'s browser \myss{b_u}, the \myss{r}'s script \myss{script\_rp} and RP's server \myss{P^r} . 
Here we only need to prove that attacker cannot receive the event from these processes carrying \myss{c(u,r)}.
\begin{itemize}
\item \myss{b.} The browsers used by users are considered honest and well implemented. Therefore, based on the same-origin policy, \myss{b_u} only sends \myss{r}'s cookie to RP's domain, so that attackers cannot receive the cookie.
\item \myss{script_rp}. According to Algorithm~\ref{alg4}, the \myss{script_rp} does not send any cookie. 
\item \myss{P^r}. According to Algorithm~\ref{alg2}, the \myss{P^r} does not send any cookie. 
\end{itemize}
Therefore, lemma~\ref{rel:cookie} is proved.
\end{proof}

To prove UPPRESSO system meets the requirement~\ref{req:cookie2}, firstly we need to know how the cookie can be set as \myss{c(u, r)}. Based on the algorithm~\ref{alg2}, we propose the new definition.
\begin{redef}
In $\mathcal{UWS}$, the cookie $c$ is to be set as\myss{c(u),r} only when RP $r$ receives a valid $u$'s identity proof, here we defined as \myss{t(u,r)}, from the owner of $c$. 
\label{red:token}
\end{redef}

To prove that \myss{t(u,r)} cannot be obtained by attackers, we firstly show some lemmas.
\begin{relemma}
Attacker does not learn users' passwords.
\label{rel:password}
\end{relemma}
\begin{proof}
Same as the proof to lemmas~\ref{rel:cookie}, we only need to prove that attackers cannot receive the message form honest processes carrying password. The honest IdP server is defined as \myss{P^i} and the IdP script is defined as \myss{script\_idp}. Here we give the proof about each processes.
\begin{itemize}
\item \myss{script\_rp}. According to algorithm~\ref{alg4}, we can prove that RP script does not send any stored passwords.
\item \myss{P^r}. According to algorithm~\ref{alg2}, it is easy to find out that RP server does not receive or send any stored passwords.
\item \myss{script\_rp}. Based on algorithm~\ref{alg3}, we can find that IdP script sends the user's password at line 68. The target of this message is \myss{Url} whose host is\myss{IdPDoaim} set at line 67. The \myss{IdPDomian} is set at line 4 and the value is defined  by the script initially with no modification. Therefore the password can only be sent to IdP server. The IdP server obtain the password at algorithm~\ref{alg1} line 10 and does not send this parameter to any other processes. 
\item \myss{P^i}. Based on algorithm~\ref{alg1},  we can find that IdP server does not send any stored passwords. 
\end{itemize}
Therefore, no attackers can obtain the password from honest processes, so that this lemma is proved.
\end{proof}

\begin{relemma}
Attacker cannot forge or modify the IdP issued proofs.
\label{rel:signature}
\end{relemma}
\begin{proof}
Here we only consider the \myss{Cert} used in \myss{script\_idp}, the \myss{RegistrationResult} and \myss{Token} used in \myss{P^r} . First of all, we can easily find that the IdP does not send the private key to any processes so that the attackers cannot obtain the private key. 
\begin{itemize}
\item \myss{Cert} is used at algorithm~\ref{alg3} line 21, 52. At line 21, the \myss{Cert} has already been verified at line 16. At line 52, the \myss{Cert} is picked from the state parameters, and the cert parameter is set at line 19.  At line 19, the \myss{Cert} has already been verified at line 16.
At line 16 the \myss{Cert} is verified with the public key in the scriptstate, where the key is considered initially honest and the key is not modified at algorithm~\ref{alg3}. Therefore, \myss{Cert} cannot be forged or modified. 
\item \myss{RegistrationResult} is used at algorithm~\ref{alg2} from line 35 to 55 after line 30 where it is verified. The public key is initially set in the RP and never modified so that  it is considered honest. Therefore, \myss{RegistrationResult} cannot be forged or modified. 
\item \myss{Token} is used at algorithm~\ref{alg2} from line 69 to 84 after line 65 where it is verified.  And the public has been proved honest. Therefore, \myss{Token} cannot be forged or modified. 
\end{itemize}
Therefore, this lemma is proved.
\end{proof}

Here we now show the lemma to prove that UPPRESSO meets the requirements in definition~\ref{red:token} .
\begin{relemma}
Attacker cannot learn users' valid identity proofs.
\end{relemma}
\begin{proof}
As the \myss{Token} has been proved that it can not be gorged by the attackers, here we only need to prove attackers cannot receive \myss{Token} from other honest processes.
\begin{itemize}
\item Attacker cannot obtain the \myss{Token} from RP server.  We check all the messages sent by the RP server at algorithm~\ref{alg2} line 4, 7, 19, 25, 31, 36, 45, 55, 61, 66, 74, 84. It is easy to proved that the RP server does not sent any \myss{Token} to other processes.
\item Attacker cannot obtain the \myss{Token} from RP script. The  messages sent by RP script can be classified into two classes. The messages at algorithm~\ref{alg4} line 18, 36, 56 are sent to the RPDomain set at line 4, so that attackers cannot receive these messages. However, the messages at line 26,  46 only carry the contents received from RP server. We have already that RP server does not send any \myss{Token}. Therefore, attackers cannot receive the \myss{Token} from RP script.
\item Attacker cannot obtain the \myss{Token} from IdP server.  Considered the messages at algorithm~\ref{alg1} line 4, 12, 16, 23, 26, 36, 44, 51, 67, it can be found that only the message at line 67 carries the \myss{Token}. This \myss{Token} is generated at line 65, following the trace where the \myss{Content} at line 63, the \myss{PID_U} at line 61, the \myss{UID} at line 60, the \myss{session} at line 48, and finally the \myss{cookie} at line 47. That is the \myss{Token} receive must be the owner of the \myss{cookie} the session of which save the parameter \myss{UID} . The \myss{UID} is set at line 15 after verifying the password and never modified. As we have already proved that the cookies and passwords cannot be known to attackers, so that attackers cannot obtain the \myss{Token} from IdP server.
\item Attacker cannot obtain the \myss{Token} from IdP script. As the proof above, only IdP sends the \myss{Token} with the message at algorithm~\ref{alg1} line 67, the IdP script can only receive the \myss{Token} at algorithm~\ref{alg3} line 99. Here we are going to proof that the token \myss{t(u,r)} can be only sent to the corresponding RP server through IdP script. The receiver of \myss{t(u,r)} is restricted by the \myss{RPOrigin} at line 100, which is set at line 55. The host in the \myss{RPOrigin} is verified included in \myss{Cert} at line 51. If the \myss{Cert} belong to \myss{r}, the attacker cannot obtain the \myss{t(u,r)}. Now we give the proof that the \myss{Cert} belong to \myss{r}. Firstly we define the negotiated \myss{PID_{RP}} in \myss{t(u,r)} as \myss{p}. That is the \myss{PID_{RP}} at algorithm~\ref{alg2} line 69 must equals with \myss{p} and the \myss{PID_{RP}} is verified at line 44 with the \myss{RegistrationToken}. This verification cannot be passed due to the state check at line 60. At the same validity period, the IdP script need to send the registration request with same \myss{p}  and receive the successful registration result. As the IdP checks the uniqueness of \myss{PID_{RP}} at algorithm~\ref{alg1} line 32. Therefore the \myss{r} and IdP script must share the same \myss{RegistrationToken}. As the \myss{RegistrationToken} contains the \myss{\mathtt{Hash}(N_U)}, the IdP script and \myss{r} must share the same \myss{ID_{RP}}. Therefore, the the \myss{Cert} saved as the IdP scriptstate parameter must belong to \myss{r}.
\end{itemize}
Therefore, attackers cannot  learn users' valid identity proofs.
\end{proof}
Here we have proved that UPPRESSO meets the requirement in definition~\ref{red:token}. Therefore the requirement~\ref{req:cookie2} is met, so that the theorem~\ref{the:secure} is proved.


\end{appendices}