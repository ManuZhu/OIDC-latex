\newc
PrivacyPass and TrustToken \cite{privacypass,trusttoken}
adopted the oblivious pseudo-random function (OPRF) protocol \cite{oprf-proved,voprf-proved,oprf-bitcoin-wallet} to generate anonymous tokens with which authorized users could access the server's resources anonymously.
To do so, a user generates a random number $e_i$ to blind an unsigned token $T_i$ into $[e_i]T_i$. After authenticating the user, the token server signs $[e_i]T_i$ with its private key $k$ and returns $[k e_i]T_i$ to the user, who converts it into an anonymous token  ($T_i, [k]T_i$) using $e_i$. 
\usso\ applies a similar cryptographic skill to design identity transformations. A user transforms $ID_{RP}$ to $PID_{RP} = [t_i]ID_{RP}$ using a random number $t_i$.
Then, the IdP calculates $PID_U = [u]PID_{RP}$, %= [ut]ID_{RP}$
where $u$ is the user's permanent identity at the IdP.
Finally, the visited RP calculates $Acct = [t_i^{-1}]PID_{U}$ using the same $t_i$ shared by the user.
%If we compare the two approaches,
The \emph{IdP-untraceability} property in \usso, i.e., $[t_i]ID_{RP}$ and $ID_{RP}$ cannot be linked by an IdP, roughly corresponds to
the {\em token signing and redemption unlinkability} in PrivacyPass/TrustToken, i.e., ($[e_i]T_i, [ke_i]T_i$) and  ($T_i, [k]T_i$) cannot be linked by the token server.

PrivacyPass/TrustToken adopted the OPRF protocol for token generation. Given a user input, the server obliviously calculates the pseudo-random output (i.e., anonymous token). The tokens are completely indistinguishable. So, if they are used to realize SSO, the RPs cannot identify each user.

In contrast, \usso\ presents a new design to leverage this cryptographic technique %in a greater depth than anonymous tokens,
in a different way, and therefore provides a non-anonymous but privacy-preserving SSO service. First, it extends the two-party OPRF protocol to work for three parties in SSO services. It not only lets the IdP to 

This enables the IdP to ``obliviously'' determine the user's account at an RP based on the user's identity and the RP's pseudo-identity (i.e., the ``blinded'' identity).
To achieve this goal, a user selects the random number $t$ to blind $ID_{RP}$ and shares $t$ with the RP to enable it to derive the user's account. %at this RP,
Moreover, the blind parameter (i.e., input to the pseudo-random functions) is utilized as the \emph{user identity} (i.e., $u$ in identity transformations) to provide sufficient flexibility required by the design. This is different from PrivacyPass/TrustToken and other OPRF-based applications \cite{privacypass,trusttoken,strong-oprf,oprf-bitcoin-wallet,pesto,oprf-ot-si,pp-ss,Private-Contact-Discovery,o-kms,oprf-deduplication} that always use it as the \emph{secret key} of the token server.%In this process, we apply pseudo-random functions but use the \emph{secret key} of pseudo-random functions as a \emph{user identity} (i.e., $u$ in identity transformations), while it always is used as \emph{secret keys} in PrivacyPass/TrustToken and other OPRF-based applications \cite{privacypass,trusttoken,strong-oprf,oprf-bitcoin-wallet,pesto,oprf-ot-si,pp-ss,Private-Contact-Discovery,o-kms,oprf-deduplication}.
This unusual application requires an in-depth understanding of the variables in OPRFs and the (pseudo-)identities in SSO services.
%This extension is unlikely to occur in OPFRs that consider $k$ as the private key of the server, but it is reasonable in \usso\ as it uses random numbers as identities and trapdoors to design oblivious pseudo-random functions for identity transformations.
%Compared with the original OPRFs, we use the $k$ key (they are always private keys) as a user identity.
% 第一段:说明2个点:
% 1. 2 party -> 3 party: user把随机数分享给RP。
% 2. 把key用作uid,在SSO中。把prf的输入输出,用作PID和account。
% 他们的token,与uid/account无关
Moreover, \usso\ leverages the randomness property of OPRFs
 to provide \emph{RP unlinkability}, while this property is not leveraged in PrivacyPass/TrustToken.
It ensures that colluding RPs cannot link any two logins across RPs, i.e., ($ID_{RP}, t, [u]ID_{RP}$) and ($ID_{RP'}, t', [u']ID_{RP'}$),
  even if they share the knowledge about the pseudo-identities and permanent accounts of the users initiating these logins.
This property results in the \emph{indistinguishability} of different users for RPs in \usso,
    therefore corresponding to different keys for signing anonymous tokens,
    which is not considered in PrivacyPass/TrustToken.
% but they intentionally distinguish different keys by configuring a unique public key at each user to verify the signed tokens.

%In addition to the extension to three parties, \usso\ leverages the OPRF cryptographic technique in greater depth than existing anonymous tokens \cite{privacypass,trusttoken}. \usso\ leverages more properties of the technique than PrivacyPass/TrustToken. The unlinkability between token signing and redemption, or ($[e_i]T_i, [ke_i]T_i$) and  ($T_i, [k]T_i$), roughly corresponds to only the IdP-untraceability in UPPRESSO: an IdP cannot link any pair of $[t_i]ID_{RP}$ and $ID_{RP}$.  % $i = 1, 2, \cdots$.
%Moreover, we prove the additional unlinkability across colluding RPs by showing that their knowledge about any two users $u$ and $u'$, i.e., ($ID_{RP}, t, [u]ID_{RP}$) and ($ID_{RP'}, t', [u']ID_{RP'}$), obtained from different logins, is indistinguishable. This property is not considered or supported in anonymous tokens \cite{privacypass,trusttoken}.
% 相比PP/TT,我们把OPRF用得更彻底,有一个新的属性。



\usso\ leverages even more properties of the OPRF protocol \cite{oprf-proved,voprf-proved}, %which are also utilized by other OPRF-based designs \cite{oprf-proved, voprf-proved, oprf-bitcoin-wallet, privacypass, trusttoken},
 to ensure its security and privacy. % in Theorems 4, 6, and 7.
It uses the \emph{obliviousness} property to prevent the IdP from learning any information about $ID_{RP}$ when receiving a token request for $[t_i]ID_{RP}$, the \emph{deterministicness} property of \emph{pseudo-random} functions to enable the RP to derive the permanent account for any $t$,
and the \emph{randomness} property to provide RP unlinkability.
 %when user identities (i.e., the key secret $k$ of pseudo-random functions) are unknown.
Last but not least, \usso\ requires an additional property for SSO services, called \emph{RP designation}.
It is satisfied only if no collision exists in RPs' pseudo-identities (see Lemma 1), which are actually the blinded inputs of the evaluated pseudo-random function.
This property   %, i.e., \emph{collision-freeness of blinded inputs},
is not explicitly required in other OPRF-based solutions and therefore may not be supported by all OPRF protocols.
Thus, an OPRF protocol is not always ready to implement identity transformations in \usso, unless no collision exists in the blinded inputs of the pseudo-random function.