\section{Threat Model and Assumptions}
\label{sec:assumptionandthreatmodel}

%Similar as other SSO systems (e.g., SAML and OIDC), UPPRESSO consists of an IdP and multiple RPs and users.
%The IdP provides user authentication services for all RPs.
%Next, we describe the threat model and some assumptions. %about these entities.

\subsection{Threat Model}
In UPPRESSO, we consider the IdP is curious-but-honest, while some users and RPs could be compromised by adversaries. % or even collude with each other.
Malicious users and RPs may behave arbitrarily or collude with each other, attempting to break the security and privacy guarantees for benign users.
%While, the IdP will follow the protocol correctly, and is only curious about the user's privacy.
%The details are as follows.

%\vspace{0.5mm}
\noindent \textbf{Curious-but-honest IdP.}
A curious-but-honest IdP strictly follows the protocol, while being interested in learning user privacy.  %without violating the protocol.
For example, it may store all the received messages to infer the relationship among $ID_U$, $ID_{RP}$, $PID_{U}$, and $PID_{RP}$ to trace a user's login activities at multiple RPs. We also assume the IdP is well-protected. %and never leaks sensitive information.
For example, the IdP is trusted to maintain the private key for signing identity proofs and RP certificates. %(see Section~\ref{implementations} for details)
So, the adversaries cannot forge an identity proof or an RP certificate.
%An honest IdP %follows the protocols to process the requests from users and RPs, and
%should not collude with malicious RPs or users,
%For example, the IdP ensures the uniqueness of $ID_{RP}$ and $ID_{U}$ when an RP or a user registers, and calculates the pseudo-identifier as the UPPRESSO protocol specifies.
%However,


%User's goal: 向IdP发送identity proof,使IdP认为自己是另一个victim
%方式:1)已经拥有有效的identity proof,希望与IdP协商出相同的PID_RP;2)通过篡改或者伪造identity proof来实现攻击

%\vspace{0.5mm}
\noindent \textbf{Malicious Users.}
We assume the adversary can control a set of users, for example by stealing users' credentials~\cite{WangZWYH16, SunCL12} or directly registering Sybil accounts at the IdP and RPs.
%These malicious users aim
%to break the security of UPPRESSO.
They may impersonate a victim user at honest RPs, or trick a victim user to log in to an honest RP under the adversary's account.
%To achieve this, they could behave arbitrarily~\cite{WangCW12, SomorovskyMSKJ12}.
For example, a malicious user may %forge the identity proof,
modify, insert, drop or replay a message, or deviate arbitrarily from the specifications when processing $ID_{RP}$, $PID_{RP}$, and identity proofs.

% the forwarding messages (requests of identity proof, identity proof,  RP registration request and result, and etc.),
%  and provide incorrect values for negotiating $PID_{RP}$ (detailed in Section~\ref{implementations}).

%RP's goal:1)获得目前登录用户在其他RP可用的identity proof;2)collusive RP 关联用户
%\vspace{0.5mm}
\noindent \textbf{Malicious RPs.}
The adversary can also control a set of RPs, for example, by directly registering at the IdP as an RP or exploiting software vulnerabilities to compromise some RPs.
The malicious RPs may behave arbitrarily to break security and privacy guarantees.
To do so, %they may attempt to obtain a valid identity proof for another RP, to allow some user to log into this target RP:
a malicious RP may manipulate its $PID_{RP}$ to trick the users to submit identity proofs generated for an honest RP to itself. %and reply them,
%when a user is logging in, to receive an identity proof that will be accepted by the target RP verifying $PID_{RP}$ but not $ID_{RP}$.
% or constructing an incorrect request to trigger the IdP issuing an identity proof binding with other RP.
%Or, the malicious RPs may collude to perform RP-based identity linkage to break user privacy.
%For example, the RPs
Or, it may manipulate its $PID_{RP}$ to affect the generation of $PID_U$ and analyze the relationship between $PID_U$ and $Account$.
%to link the user's multiple logins at different RPs. %by providing correlated values (e.g., $PID_{RP}$) to the IdP.

%\vspace{0.5mm}
\noindent \textbf{Collusive Users and RPs.} %In particular,
Malicious users and RPs may collude with each other %and behave arbitrarily,
to break the security and privacy guarantees.
For example, the adversary can first pretend to be an honest RP and trick the victim user into submitting her identity proof to it. With the valid identity proof, it can impersonate the victim user and log in to the honest RP.
%For example, malicious users and RPs may manipulate $PID_U$ and $PID_{RP}$ in an identity proof collusively to perform the impersonation or identity injection attacks.
%Or,
%    they collude to

%the adversary may first act as a malicious RP, and make an incorrect identity proof generated for the visiting user,
%  then act a malicious user, and use this identity proof to impersonate this victim user at another RP.
%The adversary could also first act as a user to login a correct RP and obtain an identity proof,
% then act a malicious RP to perform the identity injection attack, by injecting this identity proof to the session between the victim user and the correct RP with other web attacks (e.g., CSRF).


\subsection{Assumptions}
We made a few assumptions about the information and implementation of the SSO system under study. First, we consider user attributes as distinctive and indistinctive attributes, where distinctive attributes contain identifiable information about a user such as her telephone number, address, driver's license, etc. We assume the RPs cannot obtain distinctive attributes in an SSO login, since a privacy-savvy user is less likely to permit the RPs to access such information, or even does not register such information with the IdP at all. Therefore, the privacy leakage due to user re-identification is considered out of the scope of this work. Moreover, we focus only on privacy attacks enabled by SSO protocols, but not network attacks such as traffic analysis that trace a user's logins at different RPs from network traffic. Next, we assume the user agent deployed at honest users is correctly implemented so that it can transmit messages to the dedicated receivers as expected. Finally, we assume TLS is adopted to secure the communications between honest entities, and the cryptographic algorithms (such as RSA and SHA-256) and building blocks (such as random number generators) are correctly implemented.
%As we consider IdP is always honest, therefore, all the parameters provided by the IdP are assumed to be honest. All the calculations and verifications conducted by IdP are correct. Meanwhile, the collusion between RPs and IdP is not considered in this paper.




%We also assume a secure random number generator is adopted in UPPRESSO to provide the unpredictable random numbers;
%and the adopted cryptographic algorithms, including the RSA and SHA-256, are secure and implemented correctly.
%Therefore,  no one without private key can forge the signature, and the adversary fails to infer the private key during the computation.
%Moreover, we also assume the security of the discrete logarithm problem is ensured.

%In UPPRESSO, we study the RP-based identity linkage caused by a same user identifier used across different RPs. In this paper, we consider the user attributes hold by IdP can be separated as distinctive and indistinctive message, which is labelled based on whether it can be only associated with specific user.
%While the RPs may be able to re-identify a user from some distinctive user attributes, such as , we consider it out of the scope of  UPPRESSO.
%However, other attributes, such as nickname, birthday and sex can be provided to RPs with the explicit consent form user.
%Also, we focus on IdP-based login tracing attacks that are enabled by SSO protocols, but do not consider other network attacks such as traffic analysis that trace a user's logins at different RPs.

%The collusive RPs may attempt to link a user  based on the identifying attributes, such as the telephone number and credit number.
%Here, we assume that the users refuse to provide these attributes to the RPs, and the correct RPs never collect these attributes as required by privacy laws (e.g., GDPR~\cite{wachter2017counterfactual}).
%Moreover, the global network traffic analysis may be adopted to correlate the user's logins at different RPs.
%  However, UPPRESSO may integrate existing defenses to prevent this attack.