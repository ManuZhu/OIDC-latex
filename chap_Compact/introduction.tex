\section{Introduction}
\label{sec:intro}
Single sign-on (SSO) protocols such as OpenID Connect (OIDC) \cite{OpenIDConnect}, OAuth 2.0 \cite{rfc6749} and SAML \cite{SAML,SAMLIdentifier},
 are widely deployed for identity management and authentication.
 With the help of SSO,
  a user logins to a website, referred to as the \emph{relying party} (RP), using his account registered at a trusted web service,
   known as the \emph{identity provider} (IdP).
An RP delegates user identification and authentication to the IdP,
    which issues an \emph{identity token} (e.g., id token in OIDC or identity assertion in SAML) for a user to visit the RP. %after authenticating this user.
For example, in an OIDC system,
     a user sends a login request to an RP,
and this RP constructs an identity-token request with its identity (denoted as $ID_{RP}$) and redirects this request to an IdP of this system.
After authenticating the user,
 the IdP issues an identity token binding the identities of both the user and the RP (i.e., $ID_U$ and $ID_{RP}$),
    which is returned to the user and forwarded to the RP.
Finally, the RP verifies the identity token to decide whether the token holder is allowed to login or not.
So a user keeps only one credential for the IdP, instead of several credentials for different RPs.

As the comprehensive solution of identity management and authentication,
    SSO services allow an IdP to provide more user attributes in identity tokens,
        along with an authenticated user's identity.
Attributes (e.g., age, hobby, education, and nationality) are maintained at the IdP,
    and enclosed in identity tokens after the user's authorization \cite{OpenIDConnect,rfc6749}.

The wide adoption of SSO raises concerns on user privacy \cite{NIST2017draft,SPRESSO,BrowserID,maler2008venn},
 because SSO facilitates curious parties to track a user's login activities.
To issue identity tokens,
in each login instance
 an IdP is aware of when and to which RP a user attempts to login.
As a result, an honest-but-curious IdP could track all the RPs that each user has visited over time \cite{BrowserID,SPRESSO},
% This data can be further analyzed to profile users' online activities, as in other web tracking attacks.
 called the {\em IdP-based login tracing} in this paper.
%This threat by curious IdPs is also discussed by recent researches .
Meanwhile, RPs learn users identities from the received identity tokens.
If the IdP encloses an identical user identity in the tokens for a user to visit different RPs \cite{maler2008venn,Google,FirefoxAccount},
     colluding RPs could link these login instances across the RPs %and track the user's activities
      to learn his online profile \cite{maler2008venn}.
We denote this privacy risk as the {\em RP-based identity linkage}.



Privacy-preserving SSO schemes try to provide comprehensive identity management and authentication,
    while protecting user privacy \cite{maler2008venn,NIST2017draft,BrowserID,SPRESSO}.
The following features of SSO are commonly desired:
(\emph{a}) \emph{User identity at an RP},
    i.e., an identity token enables an RP to identify every user uniquely,
(\emph{b}) \emph{User authentication to only a trusted IdP}, i.e.,
    the steps of authentication between a user and the RP are eliminated,
    and a user only needs to hold the secret credential to authenticate himself to an IdP,
%    different types of credentials are supported in the authentication between the IdP and users,
and (\emph{c}) \emph{Provision of IdP-confirmed user attributes},
    i.e., a user maintains his attributes at the trusted IdP,
    and RP-requested attributes are provided %with the user's identity
            after authorized by the user.
Meanwhile,
    the privacy threats from different types of adversaries are considered:
    (\emph{a}) \emph{an honest-but-curious IdP},
    (\emph{b}) \emph{colluding RPs},
    and (\emph{c}) \emph{the honest-but-curious IdP colluding with some RPs}.
We analyze existing privacy-preserving solutions of SSO and also identity federation
in Section \ref{subsec-solutions}.


We propose {\em identity transformations} for privacy-preserving SSO,
 %and explain the reasons that limit existing solutions from fully protecting user privacy against both curious IdPs and colluding RPs.
and then design an Untraceable and Unlinkable Privacy-PREserving Single Sign-On (UPPRESSO) protocol.
 % to protect user privacy.
We design identity-transformation algorithms in the SSO login flow.
In each login instance,
        $ID_{RP}$ is transformed to an ephemeral $PID_{RP}$  by a user and an RP.
$PID_{RP}$ is then sent to an IdP to transform $ID_U$ into ephemeral $PID_U$,
    so an identity token binds $PID_U$ and $PID_{RP}$, instead of permanent $ID_U$ and $ID_{RP}$.
Finally,
    on receiving an identity token with matching $PID_{RP}$,
        the RP transforms $PID_U$ into an account.
Given a user, this account is identical across multiple login instances for an RP
     but unique at each RP.

UPPRESSO prevents the IdP-based login tracing as only $PID_{RP}$ is sent in identity-token requests,
    and the RP-based identity linkage for every user account is unique
    (see Section \ref{sec:analysis} for details).
On the contrary,
     existing privacy-preserving SSO solutions \cite{BrowserID,SPRESSO,NIST2017draft,FirefoxAccount} prevent only one of these two privacy threats.
The identity transformations work compatibly with
    the widely-used SSO protocols \cite{OpenIDConnect,rfc6749,SAML,NIST2017draft},
    so the above desirable features of SSO services are kept in UPPRESSO,
    while not all these features are supported in privacy-preserving identity federation \cite{PseudoID,ELPASSO,UnlimitID,Opaak,uprov,hyperledge-idemix}.
%
%
Our contributions are as follows.
\vspace{-\topsep}\begin{itemize}
\setlength{\topsep}{0pt}
\setlength{\partopsep}{0pt}
\setlength{\itemsep}{0pt}
\setlength{\parsep}{0pt}
\setlength{\parskip}{0pt}
\item An identity-transformation approach is proposed for privacy-preserving SSO services,
        and we design identity-transformation algorithms with desirable properties.
\item
The UPPRESSO protocol is proposed based on the identity transformations,
    with several designs specific for web applications.
We prove that UPPRESSO satisfies the security and privacy requirements of SSO services.

\item
We build the UPPRESSO prototype for web applications,
    on top of an open-source OIDC implementation.
The experimental performance evaluations show that UPPRESSO introduces reasonable overheads.
\end{itemize}


%The remainder is organized as below.
Section \ref{sec:background} presents
    the background and related works.
The identity-transformation framework is described in Section \ref{sec:challenge},
    and Section \ref{sec:UPPRESSO} presents the detailed designs.
Security and privacy are analyzed in Section \ref{sec:analysis}.
We explain the prototype implementation and evaluations in Section \ref{sec:implementation},
 and discuss extended issues in Section \ref{sec:discussion}.
Section \ref{sec:conclusion} concludes this work.
