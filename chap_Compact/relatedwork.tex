\newc

\subsection{Anonymous Tokens and OPRF-based Applications}
\label{sec:related}

PrivacyPass and TrustToken \cite{privacypass,trusttoken} adopted the oblivious pseudo-random function (OPRF) protocol \cite{oprf-proved,voprf-proved,oprf-bitcoin-wallet} to generate anonymous tokens with which authorized users could access resources anonymously.
A user generates a random number $e_i$ to blind an unsigned token $T_i$ into $[e_i]T_i$. After authenticating the user, a token server signs $[e_i]T_i$ with its private key $k$ and returns $[k e_i]T_i$ to the user, who converts it into an anonymous token  ($T_i, [k]T_i$) using $e_i$. In this process, the server obliviously calculates the pseudo-random output (i.e., anonymous token).
The RPs do not identify each user if we use these tokens to realize an SSO service, because these tokens are completely indistinguishable.

\usso\ applies a similar cryptographic skill in identity transformations  (see details in Section \ref{sec:UPPRESSO}).
 A user selects a random number $t_i$ to transform $ID_{RP}$ to $PID_{RP} = [t_i]ID_{RP}$, which hides $ID_{RP}$ from the IdP.
Then, the IdP uses the user's permanent identity $u$ to calculate $PID_U = [u]PID_{RP} = [ut_i]ID_{RP}$.
This is similar to the ``token signing'' action of the token server in PrivacyPass/TrustToken. Hence, \usso\ provides the \emph{IdP-untraceability} property, i.e., the IdP cannot link $ID_{RP}$ and $[t_i]ID_{RP}$, which roughly corresponds to the {\em token signing-redemption unlinkability} in PrivacyPass/TrustToken, i.e., the token server cannot link $T_i$ and $[e_i]T_i$ (or $[k]T_i$ and $[ke_i]T_i$). %($[e_i]T_i, [ke_i]T_i$) and  ($T_i, [k]T_i$).

Furthermore, to support SSO services, \usso\ is required to work among three parties, unlike two-party OPRF and PrivacyPass/TrustToken protocols.
It leverages this cryptographic skill in different ways, offering a non-anonymous but privacy-preserving SSO service. %, unlike the completely anonymous service in PrivacyPass/TrustToken.
%This also requires extending the two-party OPRF protocol to work for three parties in SSO services.
First, \usso\ securely shares the user-selected random number $t_i$ with the target RP,
enabling it to independently derive the user's unique account, i.e., $Acct = [t_i^{-1}]PID_{U}$. This account is ``obliviously'' determined by the IdP when it calculates the user's pseudo-identity based on the ``blinded'' input (i.e., the RP's pseudo-identity). Second, $u$ and $ID_{RP}$, which correspond to $k$ and $T_i$ in PrivacyPass/TrustToken, are designed as the \emph{permanent identities} of the user and the RP, respectively, to establish the relationships between identities in an oblivious way.
This is different from existing OPRF-based systems that always use $k$ as the server's \emph{secret key} \cite{privacypass, trusttoken, strong-oprf, oprf-bitcoin-wallet, pesto, oprf-ot-si, pp-ss, Private-Contact-Discovery, o-kms, oprf-deduplication}.
The integration of SSO (pseudo-)identities  and OPRF variables requires a deep understanding of both SSO and OPRF protocols.
Finally, \usso\ leverages the OPRF's randomness property to prove \emph{RP unlinkability}, which is not needed in PrivacyPass/TrustToken.
It ensures that colluding RPs cannot link any two logins, %i.e., ($ID_{RP}, t, [u]ID_{RP}$) and ($ID_{RP'}, t', [u']ID_{RP'}$),
even by sharing their knowledge about pseudo-identities and permanent accounts related to the logins.
This property offers desirable indistinguishability of \emph{different users} logging into the colluding RPs.
It corresponds to the indistinguishability of \emph{different keys} for signing anonymous tokens, which is not considered in PrivacyPass/TrustToken. % but they intentionally distinguish different keys by configuring a unique public key for each user to verify the signed tokens.

Although \usso\ mathematically employs the same algorithms in its identity transformations as the OPRF protocol \cite{oprf-proved,voprf-proved}, we prove more properties of these algorithms to ensure security and privacy. % in Theorems 4, 6, and 7.
\usso\ depends on the \emph{obliviousness} property to ensure IdP untraceability, %prevent the IdP from learning any information about $ID_{RP}$ when receiving a token request for $[t_i]ID_{RP}$,
the \emph{deterministicness} property of \emph{pseudo-random} functions to enable the RP to derive a permanent account for any $t$, and the \emph{randomness} property to provide RP unlinkability. %when user identities (i.e., the key secret $k$ of pseudo-random functions) are unknown.
Last but not least, \usso\ requires an additional property for SSO services, called \emph{RP designation}.
It is satisfied \emph{only if} no collision exists in RPs' pseudo-identities (see Lemma \ref{lemma-rp}), which are actually the blinded inputs of the evaluated pseudo-random function \cite{oprf-proved,voprf-proved}.
This property %, i.e., \emph{collision-freeness of blinded inputs},
is not explicitly required in other OPRF-based solutions and therefore may not be supported by all OPRF protocols. Thus, an OPRF protocol is not always ready to implement identity transformations in \usso, unless no collision exists in the blinded inputs.

\begin{comment}
PrivacyPass and TrustToken \cite{privacypass,trusttoken}
adopted the oblivious pseudo-random function (OPRF) protocol \cite{oprf-proved,voprf-proved,oprf-bitcoin-wallet} to generate anonymous tokens for authorized users to access resources. A user generates a random number $e_i$ for each unsigned token $T_i$ and blinds $T_i$ into $[e_i]T_i$.
Once the user is authenticated, a token server signs $[e_i]T_i$ with a private key $k$ as $[k e_i]T_i$. Then, the user converts $[ke_i]T_i$ to $[k]T_i$ using $e_i$ and redeems the token ($T_i, T_i^k$) to anonymously access resources. \usso\ applies a similar cryptographic skill to design identity transformations. A user transforms $ID_{RP}$ to $PID_{RP} = [t_i]ID_{RP}$ using a random number $t_i$.
Then, the IdP calculates $PID_U = [u]PID_{RP}$ %= [ut]ID_{RP}$
from $PID_{RP}$ using the user's identity $u$. Finally, the visited RP calculates $Acct = [t_i^{-1}]PID_{U}$ from $PID_{U}$ using the shared $t_i$. The \emph{IdP-untraceability} property in \usso, i.e., $[t_i]ID_{RP}$ and $ID_{RP}$ cannot be linked by an IdP, roughly corresponds to the unlinkability between token signing and redemption in PrivacyPass/TrustToken, i.e., ($[e_i]T_i, [ke_i]T_i$) and  ($T_i, [k]T_i$) cannot be linked by the token server.


%%%%%%%%%%%%%%
%%% 以下的段落,我尽量不希望写成“我们直接借鉴OPRF协议来设计”。
%% 原因:1. 的确不是这样;2. 如果我们是直接借鉴OPRF协议来设计UPPRESSO,则后面的行文不应该是现在这样子。我们后面的全文行文,完全是在谈ID transformation的角度。
PrivacyPass/TrustToken adopted the OPRF protocol for token generation, i.e., the server obliviously calculates a pseudo-random output for any user input as tokens, and the anonymous tokens are indistinguishable.
If they are used to realize SSO, the RPs cannot identify each user. In contrast, \usso\ applies this cryptographic skill in different ways from anonymous tokens, to provide a non-anonymous but privacy-preserving SSO service. First, it works for three parties in SSO while OPRFs and PrivacyPass/TrustToken are two-party protocols. In \usso\ a user shares the random number $t$ with an RP, which is used for protecting $ID_{RP}$ and enables the RP to derive the user's account, %at this RP,
and the IdP ``obliviously'' determines the account based on the user identity and the RP's pseudo-identity (or ``blinded'' identity).
In this process, we actually apply pseudo-random functions but uses the \emph{secret key} of pseudo-random functions as a \emph{user identity} (i.e., $u$ in identity transformations), while it is used as \emph{secret keys} in PrivacyPass/TrustToken and other OPRF-based applications \cite{privacypass,trusttoken,strong-oprf,oprf-bitcoin-wallet, pesto,oprf-ot-si,pp-ss, Private-Contact-Discovery,o-kms,oprf-deduplication}.
This unusual application requires a special understanding of the variables in OPRFs and the (pseudo-)identities in SSO services.
%This extension is unlikely to occur in OPFRs that consider $k$ as the private key of the server, but it is reasonable in \usso\ as it uses random numbers as identities and trapdoors to design oblivious pseudo-random functions for identity transformations.
%Compared with the original OPRFs, we use the $k$ key (they are always private keys) as a user identity.

% 第一段:说明2个点:
% 1. 2 party -> 3 party: user把随机数分享给RP。
% 2. 把key用作uid,在SSO中。把prf的输入输出,用作PID和account。
% 他们的token,与uid/account无关
Moreover, \usso\ actually leverages the randomness property of OPRFs to provide \emph{RP unlinkability}, while this property is not needed in PrivacyPass/TrustToken. It ensures colluding RPs cannot link any two logins  across RPs, i.e., ($ID_{RP}, t, [u]ID_{RP}$) and ($ID_{RP'}, t', [u']ID_{RP'}$), even if they share the knowledge about the pseudo-identities and permanent accounts of the users initiating these logins.
In \usso\ this property results in the indistinguishability of \emph{different users} in colluding RPs' view, therefore corresponding to \emph{different keys} for signing anonymous tokens, which is not considered in PrivacyPass/TrustToken. % but they intentionally distinguish different keys by configuring a unique public key for each user to verify the signed tokens.

Although \usso\ employs the same mathematical algorithms in its identity transformations as the OPRF protocol \cite{oprf-proved,voprf-proved}, we prove more properties of these algorithms to ensure security and privacy. % in Theorems 4, 6, and 7.
\usso\ depends on the \emph{obliviousness} property to ensure IdP untraceability, %prevent the IdP from learning any information about $ID_{RP}$ when receiving a token request for $[t_i]ID_{RP}$,
the \emph{deterministicness} property of \emph{pseudo-random} functions to enable the RP to derive a permanent account for any $t$, and the \emph{randomness} property to provide RP unlinkability. %when user identities (i.e., the key secret $k$ of pseudo-random functions) are unknown.
Last but not least, \usso\ requires an additional property for SSO services, called \emph{RP designation}.
It is satisfied \emph{only if} no collision exists in RPs' pseudo-identities (see Lemma 1), which are actually the blinded inputs of the evaluated pseudo-random function \cite{oprf-proved,voprf-proved}.
This property %, i.e., \emph{collision-freeness of blinded inputs},
is not explicitly required in other OPRF-based solutions and therefore may not be supported by all OPRF protocols. Thus, an OPRF protocol is not always ready to implement identity transformations in \usso, unless no collision exists in the blinded inputs.

%For example, we plan to analyze this property of quantum-secure ORPF protocols \cite{ideal-lattice-oprf,isogency-oprf} in the future, to find out whether they work effectively with the identity transformations in \usso.

%\footnote{Due to the extension of three parties and the requirement of input-collisionlessness, we did not realize this cryptographic technique has been published in OPRFs until anonymous reviewers pointed it out.}

\end{comment}

\oldc
\subsection{Extended Related Work}
%Cryptographic primitives are used for protecting user privacy.
\noindent\textbf{Cryptographic primitives for sign-on privacy.}
Distributed key generation and zero-knowledge proofs are used to generate ring-signature private keys as untraceable pseudonyms \cite{crypto-book} or prove users' statements about their credentials without revealing the credential contents \cite{zklaim}.
One-time key-share tokens \cite{tandem} are generated in two-party threshold cryptography systems to protect users' key-usage patterns.
\newc
PESTO \cite{pesto} combines distributed partially-oblivious PRFs and distributed signatures to build proactive-secure SSO services, and proposes to sign a commitment of an identity token (but not the token itself) for better user privacy.

%\vspace{0.5mm}
 \oldc\noindent\textbf{Formal analysis of SSO protocols.}
A formal analysis on SAML-based SSO \cite{ArmandoCCCT08} found that a Google-implemented variant does not bind RP identities correctly in the identity tokens.
Fett et al. \cite{FettKS16, FettKS17} formally analyzed OIDC and OAuth 2.0 using a Dolev-Yao style model \cite{FettKS14} and reported the 307 redirection and IdP mix-up attacks.
In this paper, we also developed a Dolev-Yao style model for \usso, helping to prove its security and privacy (see Section \ref{dy-model}).


%\vspace{0.5mm}
\noindent\textbf{Implementation vulnerabilities in SSO.}
Various vulnerabilities have been found in several SSO systems for web applications, resulting in attacks %of impersonation and identity injection
that break the confidentiality \cite{WangCW12,ccsSunB12, ArmandoCCCPS13, DiscoveringJCS,dimvaLiM16}, integrity \cite{WangCW12, SomorovskyMSKJ12, WangZLG16, MainkaMS16, MainkaMSW17,dimvaLiM16} or RP designation \cite{WangZLG16, MainkaMS16, MainkaMSW17, YangLCZ18,dimvaLiM16} of identity tokens.
%In the SSO services of Google and Facebook, %from the view of browser-relayed traffics
%    logic flaws of the IdPs and RPs were detected \cite{WangCW12}.  % to break the confidentiality and integrity of identity tokens.
The integrity of identity tokens was violated %\cite{SomorovskyMSKJ12,WangCW12,WangZLG16,MainkaMS16, MainkaMSW17}
due to software flaws such as defective verification by RPs \cite{WangCW12,WangZLG16,MainkaMSW17}, XML signature wrapping \cite{SomorovskyMSKJ12}, and IdP spoofing \cite{MainkaMS16,MainkaMSW17}.
Meanwhile, the RP designation was broken because of incorrect binding by an IdP \cite{YangLCZ18, WangZLG16} or insufficient verification by RPs \cite{MainkaMS16, MainkaMSW17, YangLCZ18}.
A defective IdP does not always enclose an identifiable Email address in signed tokens \cite{WangCW12},
 which breaks the user identification.
Similar vulnerabilities have been found in Android Apps that break the confidentiality \cite{ChenPCTKT14, WangZLLYLG15, YangLS17, ShiWL19}, integrity \cite{ChenPCTKT14, YangLS17}, and RP designation \cite{ChenPCTKT14, ShiWL19, WangZLLYLG15} of SSO identity tokens.

%Navas et al. \cite{NavasB19} discussed the possible attack patterns against OIDC services.
%Automatic tools such as SSOScan \cite{ZhouE14}, OAuthTester \cite{YangLLZH16} and S3KVetter \cite{YangLCZ18},
% detect the violations of confidentiality, integrity, or RP designation of SSO identity tokens.
% Wang et al. \cite{ExplicatingSDK} detect the vulnerable applications
%     built with authentication/authorization SDKs,
%      due to the implicit but unsuitable assumptions of these SDKs.


%Furthermore, if a user is compromised, the attacker can log in to RPs on his behalf. So, we consider malicious users, malicious RPs, and colluding users and RPs in our threat model (see Section \ref{subsec:threatmodel}).





% In a mobile system,
% browsers, IdP Apps,
%     or IdP-provided SDKs %(e.g., an encapsulated WebView)
%          are responsible for forwarding identity tokens, %from the IdP App to RP Apps.
% but none of them ensures an identity token is sent to the designated RP only \cite{ChenPCTKT14, WangZLLYLG15}.
% %    because a WebView or the system browser cannot authenticate the RP Apps and the IdP App may be repackaged.
% %SSO protocols are modified for mobile Apps, but the modifications are not well understood by developers \cite{ChenPCTKT14, YangLS17}.
% Vulnerabilities were found in Android Apps,
%     to break confidentiality \cite{ChenPCTKT14,WangZLLYLG15,YangLS17,ShiWL19}, integrity \cite{ChenPCTKT14,YangLS17}, and RP designation \cite{ChenPCTKT14,ShiWL19} of identity tokens.
% A flaw was found in Google Apps \cite{ArmandoCCCPS13}, allowing a malicious RP to hijack a user's authentication attempt and inject a payload to steal the cookie or identity token belonging to another RP.

% If a user is compromised,
%     attackers will log in to RPs on his behalf.
% Single sign-off helps the victim user
%  to revoke all his tokens accepted and log out from the RPs  \cite{GhasemisharifRC18}.
% FedCM \cite{FedCM} attempts to disable iframe and third-party cookies in SSO, which could be exploited to track users.
% %UPRRSSO protects privacy in SSO through ID transformations and our prototype does NOT use either iframe or third-party cookies.
