\subsection{Related Works}
\label{sec:related}

%Such tokens (or credentials) authorize a user to conduct operations
%        in privacy-preserving ways.
%
%    tokens (or credentials) authorize a user to conduct operations
%        in privacy-preserving ways.
%Privacy-enhancing technologies have been applied in various scenarios,
%  but not adopted to comprehensively transform the five (pseudo-)identities in SSO services.

\newc
\noindent\textbf{Privacy-preserving cryptographic techniques.}
PrivacyPass and TrustToken \cite{privacypass,trusttoken} design anonymous tokens for applications that grant anonymous access to authorized users, using a cryptographic technique that was originally proposed for oblivious pseudo-random functions (OPRFs) \cite{oprf-proved,voprf-proved,oprf-bitcoin-wallet}.
A user generates a random number $e_i$ for each unsigned token $T_i$ and blinds $T_i$ into $[e_i]T_i$.
After the user is authenticated, a token server signs ($[e_i]T_i, [e_i k]T_i$) with a private key $k$. Then, the user converts $[e_i k]T_i$ to $[k]T_i$ using $e_i$, and redeems the token ($T_i, T_i^k$) to access resources.
In \usso, the identity transformations are implemented using a similar cryptographic technique.
In particular, a user transforms $ID_{RP}$ to $PID_{RP} = [t]ID_{RP}$ using a random number $t$,
 and then the IdP calculates $PID_U = [u]PID_{RP} = [ut]ID_{RP}$ from $PID_{RP}$.
 The visited RP finally calculates $Acct = [u]ID_{RP}$ from $PID_{U}$ by using the shared $t$.

Anonymous tokens \cite{privacypass,trusttoken} do not identify each user at any visited RP,
    while \usso\ provides non-anonymous but privacy-preserving SSO services because we leverage the cryptographic technique in a very different way.
The OPRF protocol designed between a server and a user \cite{oprf-proved,voprf-proved,oprf-bitcoin-wallet},
    is straightforwardly adopted in PrivacyPass/TrustToken to sign anonymous tokens by a token server for users
        (i.e., the token server calculates the pseudo-random result, on receiving an input from users).
On the contrary, \usso\ extends the original two-party protocol to work for three parties in SSO:
        the random number $t$ to protect $ID_{RP}$ is shared between the user and the visited RP,
            which shall be held only by the user in the original OPRF protocol,
                the sharing of $t$ enables the RP to derive the user's permanent account at this RP,
        and the scalar $u$ plays as the user's identity known to the user and the IdP.
This extension requires an in-depth understanding of the variables in the OPRF protocol and the (pseudo-)identities of users and RPs in SSO identity tokens.

In addition to the extension to three parties,
    \usso\ exploits the OPRF cryptographic technique in greater depth than existing anonymous tokens \cite{privacypass,trusttoken}.
\usso\ leverages more properties of the technique than PrivacyPass/TrustToken.
The unlinkability between token signing and redemption, or ($[e_i]T_i, [e_ik]T_i$) and  ($T_i, [k]T_i$), roughly corresponds to only the IdP-untraceability in UPPRESSO: an IdP cannot link any pair of $[t_i]ID_{RP}$ and $ID_{RP}$.  % $i = 1, 2, \cdots$.
Moreover,
    we prove the additional unlinkability across colluding RPs by showing that their knowledge about any two users $u$ and $u'$, i.e., ($ID_{RP}, t, [u]ID_{RP}$) and ($ID_{RP'}, t', [u']ID_{RP'}$), obtained from different logins, is indistinguishable.
This property is not considered or supported in anonymous tokens \cite{privacypass,trusttoken}.

%Several cryptographic primitives have been used for protecting user privacy. For example, PrivacyPass and TrustToken \cite{privacypass,trusttoken} adapted the oblivious pseudorandom function (OPRF) protocol~\cite{oprf-proved,voprf-proved,oprf-bitcoin-wallet} to generate anonymous tokens, which grant anonymous access to authenticated users. Anonymous token generation typically involves three steps: (1) {\em token blinding}, where a user selects a random number $e_i$ to blind a token $T_i$ into $T_i^{e_i}$; (2) {\em token signing}, where the token server signs ($T_i^{e_i}, T_i^{e_i k}$) with its private key $k$ after authenticating the user; and (3) {\em token unblinding}, where the user uses $e_i$ to convert $T_i^{e_i k}$ to $T_i^k$. An unused anonymous token ($T_i, T_i^k$) can be redeemed later to access resources on the token server.

%Hence, PrivacyPass and TrustToken could be adapted to support an SSO-like service, in which the user authenticates once to a token server and redeems anonymous tokens at the server later without further authentication. However, we need to point out that PrivacyPass, TrustToken, and other OPRF protocols are two-party protocols that are not designed for SSO. The anonymous tokens are signed by and redeemed at the token server, using the server's long-term private key $k$.

%On the contrary, SSO involves three parties, where the tokens signed by the IdP are redeemed at the RPs. The IdP and the RPs cannot directly share any secrets due to the privacy requirement, therefore, applying OPRF protocols (including PrivacyPass and TrustToken) for privacy-preserving SSO requires extending the protocols from two-party to three-party, which is non-trivial. First, it requires the user to coordinate the token signing process such that the IdP uses an RP-relevant secret $k$ to sign the token without knowing the RP's identity. Besides, it requires the RPs to be able to recognize the same user in her multiple logins. Existing OPRF protocols cannot satisfy this requirement as the anonymous tokens are completely indistinguishable from each other.

More importantly, \usso\ even leverages more properties of the cryptographic technique than the OPRF protocols \cite{oprf-proved,voprf-proved,oprf-bitcoin-wallet}.
The \emph{obliviousness} property of this cryptographic technique prevents the IdP from learning anything about $ID_{RP}$ when receiving a token request for $[t]ID_{RP}$,
the \emph{deterministicness} property of \emph{pseudo-random functions} enables the derivation of permanent accounts at the RP for a random $t$,
while the \emph{randomness} property prevents colluding RPs from linking $Acct = [u]ID_{RP}$ across different RPs.
We prove that \usso\ satisfies these requirements of privacy-preserving SSO in 
    Theorems 6, 4, and 7.
Last but not least,
    \usso\ requires \emph{RP designation},
        which is ensured by another property of this cryptographic technique:
        there is no $PID_{RP}$ collision in the \usso\ protocol.
This property is proved in Lemma 1.
However, this property is \emph{not} considered in OPRF protocols;
    i.e., the OPRF protocol \cite{oprf-proved,voprf-proved,oprf-bitcoin-wallet} does not require no collision of the blinded inputs to the OPRF server.
An OPRF protocol does not always result in a variation of the \usso\ protocol,
    unless the property of input-collisionlessness is proved.\footnote{In fact, due to the extension of three parties and the requirement of input-collisionlessness, we did not realize this cryptographic technique has been published in OPRFs
        until anonymous reviewers pointed out it.}

%It shares a random number $t$ between the user and the RP to transform $ID_{RP}$ to $PID_{RP} = [t]ID_{RP}$ so that $PID_{RP}$ is related to $ID_{RP}$ but ``oblivious'' to the IdP to achieve IdP-untraceability. Then, the IdP transforms $PID_{RP}$ further to construct $PID_U=[ut]ID_{RP}$, which is ``pseudo-random'' to the RP and achieves RP-unlinkability. Finally, $PID_U$s of the same user at the same RP are associated by $[u]ID_{RP}$, which can be derived using the trapdoor $t$.
%Besides, the SSO service also requires $PID_{RP}$ to be unique (i.e., RP designation), which is not required in the original OPRF \cite{oprf-proved,voprf-proved,oprf-bitcoin-wallet}. We support this property in \usso~and prove it in Section~\ref{analysis-security}.


\begin{comment}
%The random numbers (or blinding factors) prevent the token server from linking token signing and redemption.

Anonymous tokens may be used to build anonymous authentication services, where an RP does not identify each user.
% they can be used to implement anonymous SSO services that are secure against IdP-based tracing and RP-based linkage.}
%\textcolor{blue}{They provide SSO services that are fundamentally different from the ones offered by UPPRESSO. As shown in Table~\ref{tbl:comparison-protocol}, they lack support for all three SSO features. First, the anonymous tokens used by different users or by the same user in different logins cannot be distinguished (i.e., no user identification at RPs), since they are signed blindly using the same server secret. Besides, the users are required to maintain anonymous tokens for asynchronous authentication in addition to the credentials for the IdP, similar to other privacy-preserving solutions based on anonymous credentials. Finally, PrivacyPass and TrustToken do not support selective user attribution provisioning. In contrast, UPPRESSO formalizes the privacy-preserving SSO process as two ID-transformation problems and generates unlinkable pseudoidentities and proofs to support the desired SSO features. Therefore,  UPPRESSO is  compatible with widely-adopted SSO protocols such as OIDC and SAML.}
%\textcolor{blue}{PrivacyPass and TrustToken allow a user to receive tokens \cite{privacypass,trusttoken}, each of which is denoted as ($T, T^{k}$), where $k$ is the token server's private key. These tokens are used to access resources anonymously in the future. To unlink token signing and redemption, a user generates a random number $e$ for each token, blinds $T$ into $T^{e}$, and sends it to request ($T^e, T^{ek}$) from the token server. The user then utilizes $e$ to obtain $T^k$ from $T^{ek}$, and then only ($T, T^{k}$) is redeemed to access resources. This cryptographic skill \cite{oprf-proved} is used in UPPRESSO similarly: a user transforms $ID_{RP}$ to $PID_{RP} = [t]ID_{RP}$ by a random number $t$, and $PID_{RP}$ is transformed again by an IdP to $[tu]ID_{RP}$. The visited RP calculates $Acct = [u]ID_{RP}$ from $[tu]ID_{RP}$ by using $t$ (see Table \ref{tbl:notations-protocol} for detailed descriptions of these notations).}

In \usso, the identity transformations are implemented using a similar cryptographic technique. In particular, a user transforms $ID_{RP}$ to $PID_{RP} = [t]ID_{RP}$ using a random number $t$, and then the IdP transforms $PID_{RP}$ further to $[ut]ID_{RP}$. The visited RP finally calculates $Acct = [u]ID_{RP}$ from $[ut]ID_{RP}$ by using $t$.
\usso~supports the IdP-untraceability property that prevents the IdP from linking $ID_{RP}$ and a corresponding $[t]ID_{RP}$.
This is similar to the unlinkability between token signing and redemption in PrivacyPass/TrustToken, which prevents the token server from linking ($T_i^{e_i}, T_i^{e_ik}$) and  ($T_i, T_i^k$).

However, \usso\ explores this cryptographic technique in greater depth, resulting in an additional privacy property that is not considered or supported in existing anonymous tokens \cite{privacypass,trusttoken}. %or OPRFs \cite{oprf-proved,voprf-proved}
The exponent $k$ in PrivacyPass/TrustToken is held only by the server as a consistent private key, and the random number $e$ is only known to a user. In contrast, the scalar $u$ in \usso\ denotes a user identity known to the user and the IdP, and the random number $t$ is an ephemeral trapdoor shared between the user and the RP.
%%The application of the cryptographic technique differs from PrivacyPass/TrustToken
%%This different design allows \usso\ to explore and offer additional privacy properties. %Furthermore, more privacy properties are explored in UPPRESSO.
The sharing of random numbers enables the derivation of permanent accounts in \usso,
 while we prove the additional unlinkability across colluding RPs by showing that their knowledge about any two users $u$ and $u'$, i.e., ($ID_{RP}, t, [u]ID_{RP}$) and ($ID_{RP'}, t', [u']ID_{RP'}$), obtained from different logins, is indistinguishable.

%given multiple users, e.g., identified as $u$ and $u'$, ($ID_{RP}, t, [u]ID_{RP}$) and ($ID_{RP'}, t', [u']ID_{RP'}$) are indistinguishable to colluding RPs.
%This property requires designing the identity transformations under not only ECDLP but also ECDDH assumptions (see Section~\ref{sec-:analysis}).

% \noindent\textbf{Anonymous SSO.}
% Such schemes allow authenticated users to access a service protected by an IdP,
%     without revealing their identities.
% Anonymous SSO was proposed for GSM communications \cite{ElmuftiWRR08},
%     and formalized \cite{WangWS13}.
% Privacy-preserving primitives, such as group signature, zero-knowledge proof, Chebyshev Chaotic Maps and proxy re-verification,
%      were adopted to design anonymous SSO \cite{WangWS13,HanCSTW18,Lee18,HanCSTWW20}.
% Anonymous SSO schemes work for some special applications,
%     but are unapplicable in most systems that require user identification for customized services.

\end{comment}

\oldc
%Cryptographic primitives are used for protecting user privacy.
Other cryptographic primitives such as zero-knowledge proofs are used in ZKlaims \cite{zklaim} to allow users to prove statements on the credentials issued by a trusted server without revealing the credential contents. Crypto-Book \cite{crypto-book} adopts distributed key generation to generate linkable-ring-signature private keys for users, and each key pair is used as an untraceable pseudonym. Tandem \cite{tandem} generates one-time-use key-share tokens for a two-party threshold cryptographic scheme implemented with a central server, to protect the privacy of key-usage patterns.


%two-party threshold cryptographic scheme implemented with a central server, to protect user private keys \cite{mRSA,ss-rsa}: to sign/decrypt a message, a user needs a token from the server.
%    Tandem \cite{tandem} decouples the obtaining and using of such tokens, for the privacy of key usage.

%\vspace{0.5mm}
\noindent\textbf{Formal analysis of SSO protocols.}
A formal analysis on SAML-based SSO \cite{ArmandoCCCT08} found that a Google-implemented variant does not bind RP identities correctly in the identity tokens.
Fett et al. \cite{FettKS16, FettKS17} formally analyzed OIDC and OAuth 2.0 using a Dolev-Yao-style model \cite{FettKS14} and reported the 307 redirection and IdP mix-up attacks.
In this paper, we also developed a Dolev-Yao-style model for \usso, helping to prove its security and privacy (see Section~\ref{dy-model}).


%\vspace{0.5mm}
\noindent\textbf{Implementation vulnerabilities in SSO.}
Various vulnerabilities have been found in several SSO systems for web applications, resulting in attacks %of impersonation and identity injection
that break the confidentiality \cite{WangCW12,ccsSunB12,ArmandoCCCPS13,DiscoveringJCS,dimvaLiM16}, integrity \cite{WangCW12,SomorovskyMSKJ12,WangZLG16,MainkaMS16, MainkaMSW17,dimvaLiM16} or RP designation \cite{WangZLG16,MainkaMS16,MainkaMSW17,YangLCZ18,dimvaLiM16} of identity tokens.
%In the SSO services of Google and Facebook, %from the view of browser-relayed traffics
%    logic flaws of the IdPs and RPs were detected \cite{WangCW12}.  % to break the confidentiality and integrity of identity tokens.
The integrity of identity tokens was violated %\cite{SomorovskyMSKJ12,WangCW12,WangZLG16,MainkaMS16, MainkaMSW17}
due to software flaws such as defective verification by RPs \cite{WangCW12,WangZLG16,MainkaMSW17}, XML signature wrapping \cite{SomorovskyMSKJ12}, and IdP spoofing \cite{MainkaMS16,MainkaMSW17}.
Meanwhile, the RP designation was broken because of incorrect binding by an IdP \cite{YangLCZ18,WangZLG16} or insufficient verification by RPs \cite{MainkaMS16,MainkaMSW17,YangLCZ18}.
A defective IdP does not always enclose an identifiable Email address in tokens \cite{WangCW12},
 which breaks the user identification.
Similar vulnerabilities have been found in Android Apps that break the confidentiality \cite{ChenPCTKT14,WangZLLYLG15,YangLS17,ShiWL19}, integrity \cite{ChenPCTKT14,YangLS17}, and RP designation \cite{ChenPCTKT14,ShiWL19,WangZLLYLG15} of SSO identity tokens.

%Navas et al. \cite{NavasB19} discussed the possible attack patterns against OIDC services.
%Automatic tools such as SSOScan \cite{ZhouE14}, OAuthTester \cite{YangLLZH16} and S3KVetter \cite{YangLCZ18},
% detect the violations of confidentiality, integrity, or RP designation of SSO identity tokens.
% Wang et al. \cite{ExplicatingSDK} detect the vulnerable applications
%     built with authentication/authorization SDKs,
%      due to the implicit but unsuitable assumptions of these SDKs.


%Furthermore, if a user is compromised, the attacker can login to RPs on his behalf. So, we consider malicious users, malicious RPs, and colluding users and RPs in our threat model (see Section~\ref{subsec:threatmodel}).





% In a mobile system,
% browsers, IdP Apps,
%     or IdP-provided SDKs %(e.g., an encapsulated WebView)
%          are responsible for forwarding identity tokens, %from the IdP App to RP Apps.
% but none of them ensures an identity token is sent to the designated RP only \cite{ChenPCTKT14,WangZLLYLG15}.
% %    because a WebView or the system browser cannot authenticate the RP Apps and the IdP App may be repackaged.
% %SSO protocols are modified for mobile Apps, but the modifications are not well understood by developers \cite{ChenPCTKT14,YangLS17}.
% Vulnerabilities were found in Android Apps,
%     to break confidentiality \cite{ChenPCTKT14,WangZLLYLG15,YangLS17,ShiWL19}, integrity \cite{ChenPCTKT14,YangLS17}, and RP designation \cite{ChenPCTKT14,ShiWL19} of identity tokens.
% A flaw was found in Google Apps \cite{ArmandoCCCPS13}, allowing a malicious RP to hijack a user's authentication attempt and inject a payload to steal the cookie or identity token belonging to another RP.

% If a user is compromised,
%     attackers will login to RPs on behalf of him.
% Single sign-off helps the victim user
%  to revoke all his tokens accepted and logout from the RPs  \cite{GhasemisharifRC18}.
% FedCM \cite{FedCM} attempts to disable iframe and third-party cookies in SSO, which could be exploited to track users.
% %UPRRSSO protects privacy in SSO through ID transformations and our prototype does NOT use either iframe or third-party cookies.

