\subsection{Extended Related Works}%各个方向全都加入,例如安全分析
\label{sec:related}
%Various SSO services have been designed, implemented and investigated for several years.

%Various SSO protocols have been proposed, such as OIDC, SAML, OAuth 2.0, Central Authentication Service \cite{aubry2004esup} and Kerberos \cite{Kerberos}.
%These protocols are widely adopted in Google, Facebook, Shibboleth project \cite{Shibboleth}, Java applications, etc.
%And, plenty of works have been conducted on privacy protection and security analysis for SSO systems.

%\subsection{Privacy protection for SSO systems.}

%\noindent\textbf{Privacy-preserving SSO systems.}
%%在描述Privacy-preserving SSO systems的内容中,是否需要原文中的以下内容:
%%1.BrowserID的安全问题 2.我们方案的优点
%%Privacy problems in SSO have been widely recognized that, as suggested in NIST SP800-63C \cite{NIST2017draft}, SSO systems should prevent both RP-based identity linkage and IdP-based login tracing. However, only a few solutions were proposed to protect user privacy so far. The pairwise user identifier, a simple design of function $\mathcal{F}_{ID_{U} \mapsto PID_{U}}$, has been specified in widely adopted deployed SSO standards, such as SAML \cite{SAML} and OIDC \cite{OpenIDConnect}. However, pairwise user identifier cannot prevent the IdP-based login tracing attack, as it doesn't adopt the function $\mathcal{F}_{ID_{RP} \mapsto PID_{RP}}$, therefore exposing $ID_{RP}$ to RP directly. To best of our knowledge, so far only SPRESSO \cite{SPRESSO} and BrowserID \cite{BrowserID} are proposed to prevent RP-based identity linkage by designing the individual function $\mathcal{F}_{ID_{RP} \mapsto PID_{RP}}$. However, neither SPRESSO nor BrowserID have proposed the function $\mathcal{F}_{ID_{U} \mapsto PID_{U}}$, which makes they are vulnerable to IdP-based login tracing. Moreover, there is no simple way to combine the existing function $\mathcal{F}_{ID_{U} \mapsto PID_{U}}$ and $\mathcal{F}_{ID_{RP} \mapsto PID_{RP}}$, as the dilemma is discussed in Section \ref{**} which breaks the \textbf{user identification} in SSO system.
%As suggested by NIST \cite{NIST2017draft}, SSO systems should prevent both  RP-based identity linkage and IdP-based login tracing.
%The pairwise user identifier is adopted in SAML \cite{SAMLIdentifier} and OIDC \cite{OpenIDConnect}, and only prevents RP--based identity linkage; while SPRESSO \cite{SPRESSO} and BrowserID \cite{BrowserID} only prevent  IdP-based login tracing.
%BrowserID is adopted in Persona \cite{persona} and Firefox Accounts \cite{FirefoxAccount}, however an analysis on Persona found IdP-based login tracing could still succeed \cite{FettKS14, BrowserID}.
%UPPRESSO prevents both the RP-based identity linkage and IdP-based login tracing and could be integrated into OIDC which has been formally analyzed \cite{FettKS17}.
%%Moreover, OAuth and OIDC allow users to determine the scope of attributes exposed to the RP.
%

\noindent\textbf{Privacy-Preserving Token or Credential.}
%Such tokens (or credentials) authorize a user to conduct operations
%        in privacy-preserving ways.
%
%    tokens (or credentials) authorize a user to conduct operations
%        in privacy-preserving ways.
In addition to sign-on, privacy enhancing technologies have been applied in various scenarios,
  but not adopted to comprehensively transform the five (pseudo-)identities in SSO services.
PrivacyPass and TrustToken \cite{privacypass,trusttoken} allow a user to receive a great number of \emph{anonymous} tokens.
 These tokens are used to access resources on content delivery networks,
    so a user does not interact with challenges such as CAPTCHAs.
ZKlaims \cite{zklaim} allow users to prove statements on the credentials issued by a trusted party
    using zero-knowledge proofs,
        but the credential contents are not revealed.
Crypto-Book \cite{crypto-book} coordinates servers to generate a ring-signature private key,
 and a user picks up his private key through a list of Email addresses (i.e., an anonymity set).
 Then the key pair works as an untraceable pseudonym to sign messages.
Two-party threshold schemes are implemented with a central server,
    to protect user private keys \cite{mRSA,ss-rsa}:
    to sign or decrypt a message, a user needs a token from the server.
    Tandem \cite{tandem} decouples the obtaining and using of such tokens,
for the privacy of key usage.


%\vspace{0.5mm}
\noindent\textbf{Anonymous SSO.}
Such schemes allow authenticated users to access a service protected by an IdP,
    without revealing their identities.
Anonymous SSO was proposed for GSM communications \cite{ElmuftiWRR08},
    and formalized \cite{WangWS13}.
Privacy-preserving primitives, such as group signature, zero-knowledge proof, Chebyshev Chaotic Maps and proxy re-verification,
     were adopted to design anonymous SSO \cite{WangWS13,HanCSTW18,Lee18,HanCSTWW20}.
Anonymous SSO schemes work for some applications,
    but are \emph{unapplicable} to most systems that require user identification for customized services.

%\vspace{0.5mm}
\noindent\textbf{Formal Analysis on SSO Protocols.}
%The SSO standards (e.g., SAML, OAuth, and OIDC) have been formally analyzed.
Fett et al. \cite{FettKS16, FettKS17} formally analyzed OAuth 2.0 and OIDC using a Dolev-Yao style model \cite{FettKS14},
    and presented the attacks of 307 redirection and IdP mix-up.
        %When the IdP misuses an HTTP 307 status code for redirection, the sensitive information (e.g., credentials) entered at the IdP
         %   will  be leaked to the RP through  the user's browser.
        %The IdP mix-up attack confuses the RP about which IdP is used and the victim RP sends the token to a malicious IdP,
        % which breaks the confidentiality of identity tokens.
%According to these formal proofs \cite{FettKS16, FettKS17},
%    OAuth 2.0 and OIDC are secure except these two attacks.
%UPPRESSO could be integrated into OIDC, which simplifies its security analysis.
SAML-based SSO is also analyzed \cite{ArmandoCCCT08},
    and it is found that RP identities are not correctly bound in the identity tokens of a variant designed by Google.



%\vspace{0.5mm}
\noindent\textbf{SSO Implementation Vulnerabilities.}
Vulnerabilities were found in SSO implementations for web applications,
    resulting in effective attacks %of impersonation and identity injection
     by breaking confidentiality \cite{WangCW12,ccsSunB12,ArmandoCCCPS13,DiscoveringJCS,dimvaLiM16}, integrity \cite{WangCW12,SomorovskyMSKJ12,WangZLG16,MainkaMS16, MainkaMSW17,dimvaLiM16} or RP designation \cite{WangZLG16,MainkaMS16,MainkaMSW17,YangLCZ18,dimvaLiM16} of identity tokens.
%In the SSO services of Google and Facebook, %from the view of browser-relayed traffics
%    logic flaws of the IdPs and RPs were detected \cite{WangCW12}.  % to break the confidentiality and integrity of identity tokens.
Integrity of identity tokens was violated in SSO systems  %\cite{SomorovskyMSKJ12,WangCW12,WangZLG16,MainkaMS16, MainkaMSW17}
due to software flaws such as
 defective verification by RPs \cite{WangCW12,WangZLG16,MainkaMSW17}, XML signature wrapping \cite{SomorovskyMSKJ12}, and IdP spoofing \cite{MainkaMS16,MainkaMSW17}.
RP designation is broken
    for incorrect binding by an IdP \cite{YangLCZ18,WangZLG16} or insufficient verification by RPs \cite{MainkaMS16,MainkaMSW17,YangLCZ18}.

Automatic tools such as SSOScan \cite{ZhouE14}, OAuthTester \cite{YangLLZH16} and S3KVetter \cite{YangLCZ18},
detect the violations of confidentiality, integrity, or RP designation of SSO identity tokens.
Wang et al. \cite{ExplicatingSDK} detect the vulnerable applications
    built with authentication/authorization SDKs,
     due to the implicit assumptions of these SDKs.
Navas et al. \cite{NavasB19} discussed the possible attack patterns against OIDC services.

In a mobile system,
browsers, IdP Apps,
    or IdP-provided SDKs %(e.g., an encapsulated WebView)
         are responsible for forwarding identity tokens, %from the IdP App to RP Apps.
but none of them ensures an identity token is sent to the designated RP only \cite{ChenPCTKT14,WangZLLYLG15}.
%    because a WebView or the system browser cannot authenticate the RP Apps and the IdP App may be repackaged.
%SSO protocols are modified for mobile Apps, but the modifications are not well understood by developers \cite{ChenPCTKT14,YangLS17}.
Vulnerabilities were found in Android Apps,
    to break confidentiality \cite{ChenPCTKT14,WangZLLYLG15,YangLS17,ShiWL19}, integrity \cite{ChenPCTKT14,YangLS17}, and RP designation \cite{ChenPCTKT14,ShiWL19} of identity tokens.
A flaw was found in Google Apps \cite{ArmandoCCCPS13}, allowing a malicious RP to hijack a user's authentication attempt and inject a payload to steal the cookie or identity token for another RP.

If a user is compromised,
    attackers will login to RPs on behalf of him.
Single sign-off \cite{GhasemisharifRC18} helps the victim
 to revoke all his tokens accepted and logout from the RPs.


%以下内容被拆分到上面三点内容了,但是注释内容有对论文的详细表述

%MohsenS16分析了在移动端使用webview实现SSO会导致恶意RP可以向WebView中插入JS代码获取用户的token (confidentiality),同时提供了对WebView的保护防止恶意RP获得token
%WangZLLYLG15 使用自动化工具对Android应用进行分析,总结了Android应用面临的安全问题:1 Vulnerability I (V1): Improper User-agent(使用WebView实现SSO), 2 Vulnerabili ty II (V2): Lack of Authentication(使用app间消息传递实现SSO), 3 Vulnerability III (V3): Inadequate transmission protection (网络传输缺少保护), 4 Vulnerabili ty IV (V4): Insecure secret Management (上面的 client-side logic), 5 Vulnerability V (V5): Problematical server-side validation (RP server与IdP server之间的消息传递没有受到保护,用户的信息可能被泄露,用户信息可能被篡改), 6 Vulnerabili ty VI (V6): Wrong authentication token (使用公开信息作为identity token,破坏confidentiality )
%YangLS17 使用工具分析Android 应用OAuth实现的问题:1 Untrusted Identity Token (使用server-to-server transmission保护,使用签名保护 integrity), 2 Heavy Client-Side Logic(客户端逻辑)
%ShiWL19 使用model-based的自动化工具,分析了Android应用OAuth实现的问题,包括:Access Token Replacement(替换token,破坏binding), access token Disclosure (confidentiality), code Disclosure (confidentiality),App Secret Disclosure (客户端逻辑,只曝露secret不会破坏安全性),Augmented Token Replacement (提供token与RP的绑定,但是可以被绕过, the attacker can extract the associated user information of victims from the IdP directly with either the stolen (i.e., network attacker) or obtained (i.e., malicious RP attacker) token, e.g., replaying Step 7 in Fig. 1. Consequently, the attacker can inject both the token and its corresponding user information in his own session),Profile Vulnerability(缺少用户明确授权,泄露用户的隐私数据)
% In 2016 Mohsen et al. \cite{MohsenS16} proposed the security of SSO systems implemented through WebView, one of the most important Android components, also facing the threaten of untrusted identity token transmission.
% Moreover, in 2016 Wang et al. \cite{WangZLG16} analysed the design and implementation of SSO systems for multiple platforms with the automatic testing. In 2015 Wnag et al. \cite{WangZLLYLG15}, in 2017 Yang et al. \cite{YangLS17} and in 2019 Shi et al. \cite{ShiWL19} issued the new vulnerabilities in mobile SSO systems and conducted security assessments for the top Android applications and and achieve the statistical result of these applications.

%YeBWD15在facebook的webview实现中,通过第三方应用获得facebook的cookie,使攻击者能够以受害者的身份登录facebook应用。we build a dummy app using the Facebook Login and we authorize the app with public profile permission. Then we used adb tool kit with root privilege to access to storage of the mobile phone. We successfully locate the cookies c_user and xs as well as the credential access_token. The cookies are stored in an sqlite database in mobile phone’s storage at path /data/data/<Apps package name> /databases/webviewCookiesChromium.db and the access_token is stored in an xml file at path /data/data/<App’s package name> /com.facebook.AuthorizationClient. WebViewAuthHandler.TOKEN\_STORE\_KEY.xml.
